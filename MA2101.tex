\documentclass{article}
\usepackage[english]{babel}
\usepackage[utf8]{inputenc}
\usepackage{amsmath,amssymb}
\usepackage{parskip}
\usepackage{graphicx}
\usepackage{hyperref}
\newcommand\tab[1][1cm]{\hspace*{#1}}

% Margins
\usepackage[top=2.5cm, left=3cm, right=3cm, bottom=4.0cm]{geometry}


\title{MA2101 (Linear Algebra 2)}
\author{Jia Cheng}
\date{January 2021}

\begin{document}

\maketitle

\section{Definitions and Formula}


\section{General}
Reference Book: Axler's Linear Algebra Done Right

\subsection{Well-defined functions}
I am inspired to discuss this by the section on quotient spaces.\\
For a vector space $V$, subspace $U$ of $V$, define $V/U=\{v+U : v\in V\}$.\\
Define vector addition on $V/U$ by $(v + U) + (w + U) = ((v+w) + U)$.
Define scalar multiplication on $V/U$ by $\lambda(v+U)=((\lambda v)+U)$.

One of the steps in proving that our newly defined $+$ is a binary operation on $V/U$, is to show that it is well-defined. That is, for each distinct element in the domain ($V/U$), there exists a unique image in the codomain.

For sets like integers, the representation of any element in the domain is unique, so we usually don't have to worry about well-definition of functions. However, in the case of quotient spaces, $v-v' \in U$ implies $(v+U)=(v'+U)$, even though we can have $v\neq w$. It is because of these multiple representations that we have to worry about "well-defined-ness". We need to make sure when $(v+U)=(v'+U),(w+U)=(w'+U)$, we have $(v+U)+(w+U)=(v'+U)+(w'+U)$.

\subsection{Matrices vs Linear Maps}
Other than determinants and the section on eigenvectors, MA1101R (Linear Algebra 1) is sufficiently rigorous.
The key difference between MA1101R and Axler's Linear Algebra is the approach taken to building up the theory.

An advantage of thinking in terms of linear maps rather than matrices is that linear maps are independent of our choice of bases. The same matrix can do very different things if we decide to change up the bases, but the same linear map remains invariant.

This is especially useful in abstract vector spaces. A generic vector space $V$ does not have a canonical basis (unlike the standard basis in $\mathbb{R}^n$ or polynomials $P(\mathbb{R})$. We can play around with linear maps without having the need to state a basis. (whereas we do if we want to write things in matrix form, or in simultaneous equation form)

An example of the usefulness of this is seen here: \url{https://math.stackexchange.com/questions/3749/why-do-we-care-about-dual-spaces}.
Linear functionals (linear maps that have the field as the codomain) allow the definition of a hyperplane without the need for a basis.

To quote Matt E, the author of the post:\\
So this gives a reasonable justification for introducing the elements of the dual space to $V$; they generalize the notion of linear equation in several variables from the case of $\mathbb{R}^n$ to the case of an arbitrary vector space.


\section{Matrices, A revision}
\subsection{Change of basis}
For simplicity, we consider linear operators in $\mathbb{R}^2$.

The following notations $M(T, B1, B2), M(v, B)$ (respectively matrix representation of a linear transformation $T$, matrix representation of a vector) are found in Axler's book.

Suppose we have 2 bases, the standard basis $(i, j)$ and another basis $(u,v)$. Now define the matrix $P=
\begin{bmatrix}
	M(u, (i,j)) & M(v, (i,j))
\end{bmatrix}
$
Then we will obtain the following equations.
\begin{align*}
	u=Pi\\
	v=Pj
\end{align*}
There are 2 ways to understand/interpret this equality.
\begin{enumerate}
	\item First, we "see" $P$ as purely a matrix, and we operator solely in the canonical basis of $\mathbb{R}^2$. Then this equality is simply an equality of matrices. i.e. When the standard basis vector $i$ is left multiplied by matrix $P$, we obtain the vector $u$, also with respect to the standard basis.
	\item Second, we view the equation as a change of basis. To be precise, the first equality can be written as $M(u, (i,j))=M(I, (u,v), (i,j))M(u, (u,v))$. The second equality is written similarly.
	\item Note that regardless of the viewpoint, the above 2 equalities are equalities in matrices! Clearly, $i$ and $u$ are not the same vector.
\end{enumerate}

\subsubsection{A possible area of confusion}
In this course (or at least chapter 1), Prof Brett usually represents multiplication of matrix $P$ with vector $v$ as $Pv$. Note that to be fully rigorous, we need to differentiate between an abstract vector and a matrix representation of a vector. For example, if we were to talk about a real polynomial $v$, which is a vector in $P(\mathbb{R})$, then obviously, matrix multiplication with $v$ is undefined.

So in my opinion, the more rigorous way to write things is $PM(v, B)$, where $B$ is a basis of the vector space $v$ lies in. $M(v,B)$ is that the matrix representation of $v$ w.r.t basis $B$.

Of course, in chapter 1, we do not deal with abstract vector spaces, so when Prof Brett writes $Pv$, he means that matrix $P$ is multiplied with the representation of $v$ in the standard basis of $\mathbb{R}^n$.

\subsubsection{MA1101R Recap}
Given 2 bases $B_1=(u_1,u_2),B_2=(v_1,v_2)$ of $\mathbb{R}^2$, then how do we find the change of basis matrix from $B_2$ to $B_1$, $M(I, B_1, B_2)$? (Notice the order "from $B_2$ to $B_1$". This will be explained in future.)\\
$M(I, B_1, B_2)=[M(u_1, B_2), M(u_2,B_2)]$\\
To find $M(u_i, B_2), i=1,2$, we aim to find $a,b\in \mathbb{R} s.t. av_1+bv_2=u_i$, i.e. $\begin{pmatrix}
	v_1 & v_2
\end{pmatrix}\begin{pmatrix}
	a\\
	b
\end{pmatrix}=\begin{pmatrix}
	u_i
\end{pmatrix}
$\\
Here, we did not write $M(v_1,\text{basis}),M(v_2,\text{basis}),M(u_i,\text{basis})$ within the expression, this basis is arbitrary, though ofc it has to be the same for each vector. Usually, this basis would be the canonical basis of $\mathbb{R}^2$.\\
Next, apply Gaussian Elimination to $\begin{pmatrix}
	v_1 & v_2 & | & u_1 & u_2
\end{pmatrix}$
\subsection{Markov Chains}
Given a finite set of states $E_1, \dots E_n$, let $A\in \mathbb{R}^{n,n}$ be the matrix where
\begin{align*}
	(A)_{i,j}=P(E_{i,m+1}\mid E_{j,m})
\end{align*}
where the $E_{j,m}$ denotes the event where state $E_j$ occurs on the $m$-th trial.

Some properties of the stochastic matrix:
\begin{enumerate}
	\item Sum of a column: $\sum_{1\leq i\leq n}(A)_{i,j}= \sum_{1\leq i\leq n}P(E_i\mid E_j)=1$
	\item Power of a matrix: $(A^k)_{i,j}=P(E_{i, m+k}\mid E_{j, m})$
\end{enumerate}

For instance, to use an example from lecture notes:
\begin{itemize}
	\item $P(\text{rain next day} \mid \text{rain today}) = 0.6$
	\item $P(\text{sunny next day} \mid \text{sunny today}) = 0.7$
\end{itemize}
The stochastic matrix is then 
\begin{align*}
	M=\begin{pmatrix}
		P(R_{i+1}\mid R_i) & P(R_{i+1}\mid S_i)\\
		P(S_{i+1}\mid R_i) & P(S_{i+1}\mid S_i)
	\end{pmatrix}
	=
	\begin{pmatrix}
		0.6 & 0.3\\
		0.4 & 0.7
	\end{pmatrix}
\end{align*}
Then probability theory tells us that
\begin{align*}
	P(R_{i+2}\mid R_i) &= P(R_{i+2}\mid R_{i+1}R_i)P(R_{i+1}\mid R_i)+P(R_{i+2}\mid S_{i+1}R_i)P(S_{i+1}\mid R_i)\\
	&= P(R_{i+2}\mid R_{i+1})P(R_{i+1}\mid R_i)+P(R_{i+2}\mid S_{i+1})P(S_{i+1}\mid R_i)=(M^2)_{1,1}
\end{align*}
The second equality is due to the assumption that whether it is rainy/sunny  on day $i+2$ depends only on whether it rains/is sunny on day $i+1$.

To calculate the actual probabilities:
Define $x_n=\begin{pmatrix}
	P(R_n)\\
	P(S_n)
\end{pmatrix}
,\, \forall n\in \mathbb{N}$\\
Then \begin{align*}
	M^kx_0&=\begin{pmatrix}
		P(R_{k}\mid R_0) & P(R_{k}\mid S_0)\\
		P(S_{k}\mid R_0) & P(S_{k}\mid S_0)
	\end{pmatrix}\begin{pmatrix}
		P(R_0)\\
		P(S_0)
	\end{pmatrix}\\
	&=P(R_0)\begin{pmatrix}
		P(R_{k}\mid R_0)\\
		P(S_{k}\mid R_0)
	\end{pmatrix}+
	P(S_0)\begin{pmatrix}
		P(R_{k}\mid S_0)\\
		P(S_{k}\mid S_0)
	\end{pmatrix}\\
	&=\begin{pmatrix}
		P(R_{k}\cap R_0)\\
		P(S_{k}\cap R_0)
	\end{pmatrix}+
	\begin{pmatrix}
		P(R_{k}\cap S_0)\\
		P(S_{k}\cap S_0)
	\end{pmatrix}\\
	&=x_k
\end{align*}
\subsection{Sums and products of eigenvalues}
We claim that trace $tr$ and determinant $det$ are attributes of linear operators, regardless of the choice of basis.
\begin{align*}
&tr(P^{-1}AP)=tr(APP^{-1})=tr(A)\\
&det(P^{-1}AP)=det(P^{-1})det(A)det(P)=det(P^{-1}P)det(A)=det(A)
\end{align*}
Hence, if $P$ is the change of basis matrix from eigenvector-basis to standard basis, and $A$ is the matrix of $T$ w.r.t standard basis, we then have 
\begin{align*}
	tr(D) &= tr(A)\\
	det(D) &= det(A)
\end{align*} 
where $D$ is a diagonal matrix (matrix of $T$ w.r.t eigenvector-basis). This says that $tr(A)$ is the sum of eigenvalues of linear map $T$, and $det(A)$ is the product of eigenvalues of linear map $T$.

Geometrically (in 2 or 3 dimensions), we can then understand the magnitude of the determinant of a linear map $T$ by the scaling of the area/volume of the parallelogram/parallelepiped bordered by linearly independent eigenvectors after applying linear transformation $T$ on those eigenvectors. Prof Brett calls this the "unit box".

\subsection{Useful formula}
This result can be proven by expanding the exponential into a sum and applying multivariable calculus. (Simply think of a matrix as a vector)
\begin{align*}
	\frac{d}{dt} e^{Bt}=Be^{Bt}
\end{align*}
Note: Here, $Bt=tB$ (multiplying constant matrix $B$ by scalar $t$)

Let $D=\begin{pmatrix}
	\lambda_1 & 0\\
	0 & \lambda_2
\end{pmatrix}$.
Then,
\begin{align*}
	e^{D}=\begin{pmatrix}
		e^{\lambda_1} & 0\\
		0 & e^{\lambda_2}
	\end{pmatrix}, 
	e^{Dt}=\begin{pmatrix}
		e^{\lambda_1t} & 0\\
		0 & e^{\lambda_2t}
	\end{pmatrix}
\end{align*}
where $t$ is a scalar.


\section{Vector spaces}
In the following chapter, Prof Brett diverges quite significantly from Axler's book. However, what this shows is that Mathematics truly is a "web" as Prof Han Fei says, instead of a series of linear progressions. i.e. There are many places/foundations that we can start from. In the following discussion, I hope to highlight some of the more notable differences from the book, as well as draw some attention to new concepts not discussed in the book.

\subsection{Direct Sums}
Direct sums are analogous to bases. Whereas a basis allows for the unique representation of vectors in terms of the spans of the basis vectors, a direct sum of vector space allows for the unique representation of a vector by choosing one vector from each vector space.

Let $U, V$ be 2 vector subspaces of some vector space $W$.
The following statements are equivalent.
\begin{enumerate}
	\item $u_1 + v_1 = u_2 + v_2 \implies u_1=u_2 \land v_1=v_2$
	\item $u + v = 0\implies u=v=0$
	\item $U\cap V=\{0\}$
\end{enumerate}
Hence any of the 3 statements can be used to define $U\oplus V$. In the case of MA2101, no. 3 is used.

\subsubsection{Multiple direct sums}
Prof Brett defines
\begin{align*}
	\oplus_{i=1}^n U_i = (\dots((U_1 \oplus U_2)\oplus U_3) \oplus \dots)\oplus U_n
\end{align*}
In particular, $U_1 \oplus U_2\oplus U_3=(U_1 \oplus U_2)\oplus U_3$

It turns out that it doesn't matter which order to conduct the direct sum.

The following definition of a multiple direct sum (i.e. direct sum of more than 2 vector spaces) is perhaps more enlightening.\\
Suppose $\sum_{i=1}^n U_i = \oplus_{i=1}^n U_i$.\\
Then $\sum_{i=1}^n u_i=0\implies \forall 1\leq i\leq n, u_i=0$

With this definition in mind, suppose $(U_1 \oplus U_2)\oplus U_3$, and let $u_1 + u_2 + u_3 = 0$. This implies $(u_1 + u_2) + u_3 = 0$, where $u_1+u_2\in U_1+U_2, u_3\in U_3$. Definition of direct sum for 2 vector spaces says that $u_1+u_2=u_3=0$.\\
Now, this reduces to $u_1+u_2=0$. Since $U_1 \oplus U_2$, we then have $u_1=u_2=0$.

Hence, it makes sense to define multiple direct sums in terms of "single" direct sums (i.e. direct sums of 2 vector spaces).

\subsection{Vector space isomorphisms}
\textbf{Definition} A vector space homomorphism is a linear transformation.

A vector space isomorphism is a bijective linear map.

It is here that the notion of finite-dimensionality is introduced. Note that it is perfectly fine to define finite-dimensional without dimension. We simply need to capture the idea of finiteness.\\
In Axler's book, a finite-dimensional vector space $V$ is one which has a finite spanning set. i.e. A finite subset of $V$ exists that spans $V$.

Prof Brett defines finite-dimension via isomorphisms. A finite-dimensional vector space $V$ is one that is isomorphic to $\mathbb{F}^n$ for some $n\in \mathbb{N}$.\\
Contrary to the book's definition, it is immediately (intuitively) obvious that such an $n$ is unique, if it exists. Suppose not, such that $V$ is isomorphic to both $F^m$ and $F^n$, $m\neq n$. Then since the composition of bijective linear maps gives another bijective linear map, $\mathbb{F}^m\cong \mathbb{F}^n$, which immediately seems absurd. (The rigorous proof for this however, probably still requires the Replacement Theorem. So on a rigorous basis, this probably isn't much easier than Axler's approach.)


\subsection{Bases as isomorphisms from $\mathbb{F}^n$ to $V$}
In this course, the convention is to use basis vectors starting from the end of the alphabet. Occasionally, I will also use $u,v$.

Given a basis $z=(z_1,\dots,z_n)$ of $V$, we can also define the mapping 
\begin{align*}
z:\mathbb{F}^n\rightarrow V\\
z(a)=\sum_{i=1}^na_iz_i
\end{align*}
where $a=(a_1,\dots, a_n)\in \mathbb{F}^n$. It is easy to verify that $z$ is a bijective linear map, i.e., $z$ is a vector space isomorphism.

\textbf{Proposition} The range/image of a linear transformation $T:U\rightarrow V$ is the span of $T$ applied to the basis vectors of $U$.

\textbf{Proposition} A vector space isomorphism $\phi: \mathbb{F}^n\rightarrow V$ admits a basis $(\phi(e_1),\dots, \phi(e_n))$.

\textbf{Proof} Since linear map $\phi$ is injective, the vectors $(\phi(e_1),\dots, \phi(e_n))$ span $V$. Since $\phi$ is injective, $\sum_{i=0}^n a_i\phi(e_i)=0\implies \phi(\sum_{i=0}^n a_ie_i)=0 \implies \sum_{i=0}^n a_ie_i=0 \implies \forall i, a_i = 0$, which proves linear independence. Hence we have a basis.

Hence, "there is a one-one correspondence between isomorphisms of the form $\mathbb{F}^n\rightarrow V$ and bases".

Here is another way to think about a basis $z=(z_1,\dots,z_n)$ of $V$ . In Axler's book, the matrix representation of a vector $v\in V$ with respect to a basis $z$ is denoted as $M(v, z)$. So if we define the linear map $M(\cdot, z): V\rightarrow \mathbb{F}^{n,1}\cong \mathbb{F}^n$, we then notice that this is the inverse function of $z:\mathbb{F}^n\rightarrow V$.


\subsection{Dimension}
In Axler's book, dimension of a vector space is defined to be the number of vectors in a basis of $V$. This is well-defined since by the Replacement Theorem, the number of vectors in any basis of a vector space is the same.

In MA2101, as mentioned previously, defining a vector space $V$ to be  finite-dimensional if it is isomorphic to $\mathbb{F}^n$ for some $n\in \mathbb{N}$ suggests a natural definition for dimension -- the value of $n$. (Of course, as mentioned previously, a rigorous treatment of this would still require proving that this $n$ is unique if it exists.) Now, we take this for granted, that $n$ is unique.

\textbf{Proposition} The dimension of a finite-dimensional vector space $V$ is equal to the number of vectors in any basis.

\textbf{Proof} Consider a basis $z=(z_1,\dots,z_n)$ of $V$. $z$ then corresponds to some vector space isomorphism $\phi_z:\mathbb{F}^n\rightarrow V$ as defined above. (The vectors of basis $z$ are then given by $\phi_z(e_1),\dots,\phi_z(e_n)$.) By definition of dimension, this precisely says that the dimension of $V$ is $n$.

\section{Linear maps}

\subsection{Change of basis}
As mentioned in the previous section, for any basis $z:\mathbb{F}^n\rightarrow V$, we have $z^{-1}=M(\cdot,z):V\rightarrow \mathbb{F}^n$.

Let $y,z$ be 2 bases of vector space $V$.\\
Let $P=z^{-1}\circ y:\mathbb{F}^n\rightarrow \mathbb{F}^n$.\\
We then have $y=z\circ P$. Here, $P$ is a change of basis from $z$ to $y$ (\textbf{\textit{note the order!}}). In the notation of Axler's book, $P=M(I, y, z)$. ($I$ is the identity linear map.) For any $v\in V$, $M(I,y,z)M(v,y)=M(v,z)$.

Another way to remember the bolded part above is to think of the change of basis of a linear operator $T\in L(V)$.
\begin{align*}
	M(T, y, y) = M(I, z, y)M(T, z, z)M(I, y, z) = P^{-1}M(T, z, z)P
\end{align*}
Hence $P$ changes the basis on which $T$ acts from $z$ to $y$.

Given a certain basis $z$ of $V$, every other basis can be expressed in the form of $z\circ P$ for some isomorphism $P:\mathbb{F}^n\rightarrow \mathbb{F}^n$.

\subsection{Dual spaces}
Given a vector space $V$ over $\mathbb{F}$. The dual space of $V$, the set of all linear functionals on $V$, is denoted $V'=\hat{V}=L(V, \mathbb{F})$.

\subsection{Kernel and range}
In MA2101, the null space/kernel of a linear map $T:V\rightarrow W$ is denoted $ker(T)$. $dim\, ker(T) = null(T)$

The range of $T$ is denoted $rang(T)$. $dim\, rang(T)=rank(T)$

In Axler's book, rank is actually defined in terms of the column/row rank of a matrix representation of $T$. Then, rank is proven to be equal to $dim\, rang(T)$ by establishing an isomorphism between the column space of $M(T,B1,B2)$ and the range of $T$.

\subsubsection{Fundamental Theorem of Linear Maps}
Given linear map $T:V\rightarrow W$,
\begin{align*}
	dim V = null(T) + rank(T)
\end{align*}

There are 2 ways to prove the dimension theorem.
The first way (Axler) is to construct a basis $B_{null(T)}=\{y_1,\dots, y_k\}$ of $ker(T)$, then extend it to a basis $B_V=\{y_1,\dots,y_k, z_1,\dots, z_m\}$ of $V$. Show that $(Tz_1,\dots, Tz_m)$ is a basis for $range(T)$. 
Hence,
\begin{align*}
	dim(V) = k + m = null(T) + rank(T)
\end{align*}

The second way (MA2101) is to find a subspace $U$ of $V$ such that $V=null(T)\oplus U$. It is possible to find such a subspace via basis extension. Show that $U\cong range(T)$ by considering the restriction of $T$ to $U$ (also restrict codomain $V$ to $range(T)$. 
\begin{align*}
	T|_U:U\rightarrow range(T)
\end{align*}
Since isomorphism by definition (in MA2101) implies equal dimension, we are done.

\subsection{Dual vectors}
\textbf{Proposition} Given a basis $z=(z_1,\dots,z_n)$ of a vector space $V$, and any choice of vectors $w_1,\dots, w_n\in W$, a linear map $T:V\rightarrow W$ is uniquely identified by $T(z_i)=w_i, i\in \{1,\dots,n\}$.

The proof is trivial.

\textbf{Corollary} Given a basis $z=(z_1,\dots,z_n)$ of a vector space $V$, and linear maps $T_1,T_2:V\rightarrow W$. To check if $T_1=T_2$, it suffices to check that $\forall i\in \{1,\dots,n\}, T_1(z_i)=T_2(z_i)$

Hence we can define dual basis vectors as follows.\\
\textbf{Definition} $\forall i,j\in \{1,\dots, n\}, \zeta_i(z_j)=\delta_{ij}$.

Prof Brett uses Einstein's notation, however, I decide to stick with the classical subscript notation. However, I will still follow certain conventions for this course, e.g. scalars start from the start of the alphabet; basis vectors start from the end of the alphabet.

To show that $\zeta$ is indeed a basis for $\hat{V}$, we need to check for linear independence and span. Linear independence is easy. As for span, for any $\alpha \in \hat{V}$, define $\beta=\sum_i(\alpha z_i)\zeta_i$. Using the corollary above, it suffices to check for $\forall j\in \{1,\dots,n\}, \alpha z_j=\beta z_j$. Hence $\alpha=\beta$ and $\zeta$ spans $\hat{V}$.

\textbf{Proposition} $\alpha\sum_ip_i\zeta_i\iff p_i=\alpha z_i$

The backward direction is proven above. The forward direction is proven by applying $\alpha$ to $z_j$ for each $j=1,\dots,n$.

Now, temporarily forget the original set-based definition of the dual basis. We want to see if we can define the dual basis using an isomorphism, i.e. a map-based definition.

So now we don't have a general definition of dual basis. However, we still wish to define a special case for the dual space of $\mathbb{F}^n, \hat{\mathbb{F}^n}$. The dual basis vectors of $\hat{\mathbb{F}^n}$ are the row vectors $(\epsilon_1,\dots \epsilon_n)$, where $\epsilon_i=(e_i)^T$.

Based on this special case, we now define the following:\\
\textbf{Definition} Given a basis $z$ of $V$, the dual basis $\zeta$ of $\hat{V}$ is the isomorphism $\zeta: \hat{\mathbb{F}^n}\rightarrow \hat{V}$ such that 
\begin{align*}
	\forall a\in \mathbb{F}^n,\, \forall p\in \hat{\mathbb{F}^n}, (\zeta p)(za)=pa
\end{align*}
 
Now, I want to recover the original set-based definition of a dual basis. Let $p=\epsilon_i, a=e_j$, then $pa=\delta_{ij}$.\\
By definition $(\zeta \epsilon_i)(z_j)=\delta_{ij}$. Since $\zeta$ is an isomorphism, it corresponds to a set-basis with $(\zeta_1,\dots,\zeta_n)$, where $\zeta_i=\zeta \epsilon_i$. Hence, $\zeta_iz_j=\delta{ij}$.\\
And we have recovered the original set definition of a dual basis.

\subsubsection{Vector acting on dual vector}
I introduce the following notation:\\
Given $v\in V, \alpha \in \hat{V}$,
\begin{align*}
	\alpha(v)=\alpha v=v[\alpha]
\end{align*}
In MA2101, Prof Brett does not use square brackets, instead just writing $v(\alpha)=v\alpha$, however I find this to be potentially confusing.\\
The motivation for writing this will come later (e.g. when showing that the matrix representation of the transpose/dual map $\hat{T}$ of $T$ is really the transpose matrix of $T$).

\textbf{Definition} Given $V, W$, $T\in L(V, W)$, the \textbf{dual map}(Axler)/\textbf{transpose}(MA2101) of $T$ is denoted by $\hat{T}\in L(\hat{W},\hat{V})$. 
\begin{align*}
	\forall \alpha \in \hat{W}, \hat{T}(\alpha)=\alpha \circ T
\end{align*}
Hence, $\forall v\in V, v[\hat{T}\alpha]=(\alpha \circ T)v=\alpha Tv$. Note that the convention of linear maps is to forego the use of round brackets.\\
Notice the symmetry here.

\subsection{Tensor products}
\textbf{Definition} The tensor product $\otimes$ is a bilinear map defined as follows:
\begin{align*}
	\otimes: &V \times \hat{V}\rightarrow L(V)\\
	&(v, \alpha) \mapsto \alpha(\cdot)v
\end{align*}
That is, for each $w\in V$, $(v\otimes \alpha)w=\alpha(w)v$, multiplying $v$ by scalar $\alpha w$.

The transpose/dual map of $T=(v\otimes \alpha)$ has the following "nice" property. \\
For all $\beta \in \hat{V}, \hat{T}\beta=\beta \circ (v\otimes \alpha)\in \hat{V}$. Applying this to any arbitrary $w\in V$, we get 
\begin{align*}
	(\beta \circ (v\otimes \alpha))w&=\beta(\alpha(w)v)\\
	&=\alpha(w)\beta(v) && \text{linearity of $\beta$}\\
	&=\beta(v)\alpha(w) && \text{commutativity of scalar multiplication}\\
	&=(\beta(v)\alpha)(w) && \text{scalar multiplication of functions}
\end{align*}
Hence, $\hat{T}\beta=(\beta \circ (v\otimes \alpha))=\beta(v)\alpha$

\subsubsection{Basis of $L(V)$}
We are given that $z=(z_i)_{1\leq i\leq n}$ is a basis of $V$.\\
Let $\zeta=(\zeta_i)_{1\leq i\leq n}$ be the dual basis of $\hat{V}$.\\
Consider the following set 
\begin{align*}
	B=\{z_i\otimes \zeta_j : i,j\in \{1,\dots, n\}\}
\end{align*}
We claim that $B$ is a basis for $L(V)$.

\textbf{Claim}: $B$ spans $L(V)$.

Let $T\in L(V)$.\\
Define the scalars $T_{i,j}, i,j\in \{1,\dots,n\}$ as follows: For each $1\leq j\leq n$, $T(z_j)=\sum_{1\leq i\leq n}T_{i,j}z_i$.

Define 
\begin{align*}
	S=\sum_{1\leq i,j\leq n}T_{i,j}z_i \otimes \zeta_j
\end{align*}
Then for each $1\leq k\leq n$, 
\begin{align*}
	S(z_k)&=\sum_{1\leq i,j\leq n}T_{i,j}\zeta_j(z_k)z_i \\
	&=\sum_{1\leq i,j\leq n}T_{i,j}\delta_{jk}z_i\\
	&=\sum_{1\leq i\leq n}T_{i,k}z_i && \text{Substitute $j=k$}\\
	&=Tz_k && \text{definition of $T_{ij}$}
\end{align*}

Since linear maps are uniquely determined by their actions on basis vectors, $S=T$. This says that $B$ spans $L(V)$.

\textbf{Claim}: $B$ is linearly independent.

Suppose there exists $\lambda_{i,j}, 1\leq i,j\leq n$ such that \begin{align*}
	\sum_{1\leq i,j\leq n}\lambda_{i,j}z_i \otimes \zeta_j=0_{L(V)}
\end{align*}
Then, for each $1\leq k\leq n$, 
\begin{align*}
	0_V=0_{L(V)}(z_k)&=\sum_{1\leq i,j\leq n}\lambda_{i,j}\zeta_j(z_k)z_i \\
	&=\sum_{1\leq i,j\leq n}\lambda_{i,j}\delta_{jk}z_i\\
	&=\sum_{1\leq i\leq n}\lambda_{i,k}z_i \\
\end{align*}
Since $z$ is a basis, it is a linearly independent set, hence $\forall 1\leq i\leq n,\, \lambda_{i,k}=0$. $k$ runs from $1$ to $n$, so we have $\lambda_{i,k}=0$ for all $1\leq i,k\leq n$. This proves linear independence of $B$. 

\textbf{Definition} $T_{i,j}$ are called the components of $T\in L(V)$ with respect to basis $z$.

Before, the definition of $T_{i,j}$ is implicitly given by $T(z_j)=\sum_{1\leq i\leq n}T_{i,j}z_i$. Here is a more explicit way to express an individual entry of $T_{i,j}$.
\begin{align*}
	\forall 1\leq i,j\leq n,\, T_{i,j}=\zeta_{i}(Tz_j)
\end{align*}

\subsection{Matrices}
In MA2101, the $(i,j)$-th entry of a $n$ times $m$ matrix $M\in \mathbb{F}^{n,m}$ is denoted as $M^i_j$.
I will use the "classical" $M_{i,j}$ instead.\\
If the entries of a matrix $M_{i,j}$ are known and equal $a_{i,j}$, then $M$ can be denoted as $(a_{i,j})$.

All the matrix notation here onwards will largely follow Axler's book.

\textbf{Proposition} Given $M\in \mathbb{F}^{n,m}, N\in \mathbb{F}^{m,p}$,
\begin{align*}
	MN = \sum_{1\leq i\leq m}M_{\cdot,i}N_{i,\cdot}
\end{align*}

This is covered in MA1101R and is one of the multiple ways with which to view matrix multiplication.

Given a linear operator $T\in L(V)$, we know from the previous section that $T=\sum_{1\leq i,j\leq n}T_{i,j}(z_i\otimes \zeta_j)$.
Its matrix representation with respect to the basis $z$ is $(T_{i,j})=(\zeta_i(Tz_j))$.

\textbf{Theorem} The matrix representation of $\hat{T}$ with respect to basis $\zeta$ is the transpose of the matrix representation of $T$ w.r.t basis $z$.

\textbf{Remark} Prof Brett actually refers to the matrix representation of $\hat{T}$ relative to basis $z$, since $\zeta$ is derived from $z$. However, personally, I'd like to emphasise that $\hat{T}$ maps from $\hat{V}$ to $\hat{V}$, so when referring to the basis I also would refer to the dual basis of $\hat{V}$.

\textbf{Proof}
We have
(1)
\begin{align*}
	T_{i,j}=\zeta_i(Tz_j)=(\zeta_i\circ T)z_j=(\hat{T}\zeta_i)z_j=z_j[\hat{T}\zeta_i]
\end{align*}
Notice that as $\hat{T}$ is just another linear operator, we have already defined $\hat{T}_{i,j}$ implicitly in the previous section as 
\begin{align*}
	\hat{T}\zeta_j=\sum_{1\leq i\leq n}\hat{T}_{i,j}\zeta_i
\end{align*}
Hence, 
(2)
\begin{align*}
	z_k[\hat{T}\zeta_i]&=\sum_{1\leq i\leq n}\hat{T}_{i,j}\zeta_i(z_k)\\
	&=\hat{T}_{k,j}
\end{align*}
Hence, combining (1) and (2), $\forall 1\leq i,j\leq n$,
\begin{align*}
	T_{i,j}=z_j[\hat{T}\zeta_i]=\hat{T}_{j,i}
\end{align*}

Hence, in matrix form, $(\hat{T}_{i,j})=(T_{i,j})^t$.


\subsection{Linear maps $L(V,W), V\neq W$}
As a convention, let $dim(V)=n, dim(W)=m$

Define the tensor product:
\begin{align*}
	\otimes: &W \times V' \rightarrow L(V,W)\\
	&(w, \alpha) \mapsto \alpha(\cdot)w
\end{align*}
i.e. $\forall v\in V, (w \otimes \alpha)v = \alpha(v)w$

This definition makes a lot more sense when we try to define the basis of $L(V,W)$ in terms of the tensor product.

Let $z=(z_1,\dots, z_n)$ be a basis for $V$.
Let $x=(x_1,\dots, x_m)$ be a basis for $W$.

\textbf{Proposition} The set 
\begin{align*}
	\{x_i \otimes \zeta_j : 1\leq i\leq m, 1\leq j\leq n\}
\end{align*}
is a basis for $L(V,W)$.

\subsection{Change of Basis}
For simplicity, we first consider a single vector space $V$.\\
Let $z,y: \mathbb{F}^n\rightarrow V$ be 2 bases of $V$, and $\zeta, \gamma$ be the corresponding dual bases.
A change of basis affects the matrix representation of the following 3 items:
\begin{enumerate}
	\item Linear maps $T\in L(V)$
	\item Vectors $v\in V$
	\item Dual vectors $\alpha \in \hat{V}$
\end{enumerate}

Define $P=z^{-1}\circ y$.

\textbf{Lemma 1} $M(I, y, z) = M(P, e, e)$\\
For convenience, we write $M(P, e) = M(P,e,e)$, since the two bases are the same.

\textbf{Proof} Consider the $(i,j)$-th entry of the matrix of $I$ w.r.t bases $y, z$. 
\begin{align*}
	M(I, y, z)_{i,j}&=\zeta_iIy_j=(\epsilon_iz^{-1})I(ye_j)\\
	&=\epsilon_i(z^{-1}y)e_j=\epsilon_iPe_j=M(P, e)_{i,j}
\end{align*}
Since all the entries of the 2 matrices are the same, we have the desired result.
\textbf{Remark} A similar derivation gives $M(I, z, y) = M(P^{-1}, e)$

\subsubsection{Linear maps}
I will present 3 different derivations. The first 2 approaches also make use of the result the matrix of composite linear maps is the product of their matrices. Comparing \textbf{1} with \textbf{2}, they both make use of the same ideas, but the first is shorter.

\textbf{1 Axler}
\begin{align*}
	M(T, y) = M(I, z, y)M(T, z)M(I, y, z) = M(P^{-1}, e)M(T, z)M(P, e)
\end{align*}
The last equality is from the use of the lemma above.

\textbf{2 MA2101 Method 1}
Note that $M(T,z)_{i,j}=\zeta_iTz_j = \epsilon_iz^{-1}Tze_j=M(z^{-1}Tz, e)$. We can similarly replace $z$ with $y$.
\begin{align*}
	M(T, y) = M(y^{-1}Ty, e) = M((zP)^{-1}T(zP),e) = M(P^{-1}(z{-1}Tz)P,e) =  M(P^{-1}, e)M(T, z)M(P, e)
\end{align*}

\textbf{3 MA2101 Method 2}
See below. This will be presented separately.

\subsubsection{Vectors}
Let $v\in V$. Define the matrix of $v$ with respect to a basis $z$ as $M(v, z)$. Clearly, $M(v, z)=M(z^{-1}v, e)$

For simplicity/laziness, I will identify the column matrix $M(z^{-1}v, e)\in \mathbb{F}^{n,1}$ as just the column vector $z^{-1}v\in \mathbb{F}^n$. Technically they are different mathematical objects.\\
I also implicitly make use of the result that $M(Tu, basis)=M(T, basis)M(u, basis)$

Then $M(v, y) = y^{-1}v = (z\circ P)^{-1}v = P^{-1}z^{-1}v = M(P^{-1}, e)M(v, z)$



\subsubsection{Dual vectors}
Let $\alpha \in \hat{V}$. 

Denote $e=\{1\}$ as the basis of $\mathbb{F}$ (when viewed as a vector space).\\
The dual basis of $e$ is $\epsilon=\{1\}$.\\
Denote $e'$ as the basis of $\mathbb{F}^n$.

Then $M(\alpha, y, e) = M(\alpha, z, e)M(I, y, z) = M(\alpha, z, e)M(P, e')$

Alternative derivation:\\
Notice that $1\alpha y_j = 1(\alpha y)(e'_j)$ for each $1\leq j\leq n$.\\
Hence, $M(\alpha, y, e) = M(\alpha y, e', e) = M(\alpha zP, e', e) = M(\alpha z, e', e)M(P, e', e')$

We can observe that the alternative way is rather analogous to how Prof Brett does it in \textbf{MA2101 Method 1} as stated above.

\subsubsection{MA2101 Method 2}
\textbf{Lemma 2} $\epsilon_j \circ P = \sum_{1\leq i\leq n}P_{ji}\epsilon_i$

\textbf{Proof} Clearly, $\epsilon_j \circ P\in \hat{\mathbb{F}^n}$. \\
For each $1\leq k\leq n$, $\epsilon_j \circ P e_k = \epsilon_j \sum_i P_{i,k} e_i = P_{j,k}$. Also, $\sum_{1\leq i\leq n}P_{ji}\epsilon_i (e_k) = P_{j,k}$. And we are done.

We first summarise how the basis vectors of $y,z,\gamma,\zeta$ are represented in terms of each other.
\begin{itemize}
	\item $y_j=y(e_j)=(zP)e_j=z\sum_iP_{i,j}e_i=\sum_iP_{i,j}z_i$
	\item Similarly, $z_j=\sum_iP^{-1}_{i,j}y_i$
	\item Using Lemma 2, $\gamma_j=\epsilon_j\circ y^{-1}=\epsilon_j \circ (zP)^{-1} =(\epsilon_jP^{-1})z^{-1}= (\sum_iP^{-1}_{j,i}\epsilon_i)z^{-1}=\sum_i P^{-1}_{j,i}\zeta_i$
	\item Similarly, $\zeta_j=\sum_i P_{j,i}\gamma_i$
\end{itemize}

\textbf{Vectors}\\
Let $v$ be a vector in $V$. Note that $z=y\circ P^{-1}$.\\
Denote $a_j:=M(v,z)_{j,1}$\\
$v = \sum_j a_j z_j = \sum_j a_j \sum_i P^{-1}_{i,j} y_i = \sum_i (\sum_j P^{-1}_{i,j}a_j) y_i$\\
Hence, $M(v,y)_{i,1}=\sum_j P^{-1}_{i,j}M(v,z)_{j,1}$ for each $1\leq i\leq n$. Hence, $M(v,y) = M(P^{-1}, e)M(v,z)$

We deal with dual vectors and linear maps in a similar fashion.

\section{Eigenthings}
\textbf{Definition} Given a linear operator $T$ on a vector space $V$, a subspace $U\subseteq V$ is an eigenray if it is invariant with respect to $T$ and is one-dimensional. 

\textbf{Proposition} The eigenvalue of an eigenray is unique.

This is a consequence of the linearity of a linear operator.

\textbf{Proposition} Given $n$ eigenrays $R_1,\dots,R_n$ of a vector space $V$ with pairwise different eigenvalues, and any choice of $n$ eigenvectors $v_i\in R_i$ (one from each eigenray), then the set $(v_1,\dots,v_n$) is linearly independent.

\textbf{Proof} By induction on $n$.\\ 
A special case for $n=2$ can be proven as follows. If $v_1, v_2$ are linearly dependent, then $v_2=kv_1$ for some scalar $k$. But this says that $v_2$ lies in span($v_1$)=$R_1$. It then follows directly that $R_1 = R_2$. But we know that the an eigenray only has one eigenvalue. Contradiction.

\textbf{Corollary} There are at most $dim(V)$ distinct eigenrays in a finite-dimensional vector space $V$.

This is a consequence of the fact that a linearly independent set of vectors in a finite-dimensional vector space $V$ can have at most $dim(V)$ vectors.
The proof is by Steinitz exchange/replacement lemma.\\

\subsection{Fundamental Theorem of Algebra}
We now assume the fundamental theorem of algebra without proof.

\textbf{Proposition} Every linear operator over a complex vector space $V$ has an eigenvalue.

Consider $v, Tv, \dots, T^nv$, where $n=dim(V)$. Since there are more than $n$ vectors, they cannot be linearly independent. That is, there exists scalars, $\lambda_i, i\in \{0,\dots,n\}$ such that $\sum_{0\leq i\leq n}\lambda_i T^iv=0$.

\begin{align*}
	\sum_{0\leq i\leq n}\lambda_i T^iv&=(\sum_{0\leq i\leq n}\lambda_i T^i)v\\
	&=a(T-a_1I)\dots(T-a_nI)v &&\text{by FTA and linearity of $T$}
\end{align*}

Note 1: Recall that for all functions, linear or not, we have $(f+g)h=fh+gh$. However, additionally linear maps are bilinear under composition, that is, $f(g+h)=fg+fh$. This is what allows the second equality to hold.

Note 2: It is also important that $V$ is a vector space over $\mathbb{C}$, otherwise, we would not be able to do the factorisation $a(T-a_1I)\dots(T-a_nI)$, where $a_i$ are possibly complex scalars. If $V$ was a vector space over $\mathbb{R}$, then $(T-iI)$ for instance would not be an operator over a real vector space, since $(T-iI)v=Tv-iv$ and we are trying to multiply $v$ by a complex scalar.\\
Additionally, if $V$ was a vector space over $\mathbb{R}$, all the $\lambda_i$ must also be real.

This lemma will allow us to "kickstart" our proof of the next theorem.

\subsection{Upper triangular and diagonal matrices}
\textbf{Definition} Given matrices $A,B$, say $A\equiv B$ or $A$ is similar to $B$ if there exists invertible matrix $P$ such that $A=P^{-1}BP$.

\textbf{Proposition} Given a linear operator $T\in L(V)$, where $V$ is a vector space over $\mathbb{C}$, there exists a choice of basis $y$ of $V$ such that $M(T, y)$ is upper triangular.

Note that this is equivalent to stating that given any basis $z$ of $V$, there exists $P$ such that $M(T, y)=M(P^{-1},e)M(T, z)M(P, e)$. i.e. $M(T, z)$ is similar to some upper triangular matrix $M(T, y)$.


(For simplicity, we will denote $M(T, y, y)$ with $M(T, y)$.)

Before proceeding with the proof, we need to recall a few things. \begin{itemize}
	\item By choosing any 2 bases $y,z$ of a vector space $V$, $P:=z^{-1}\circ y$ is a change of basis map. $P$ is the composition of 2 isomorphisms, so it is itself an isomorphism. 
	\item We choose one basis $z$ of $V$ and some isomorphism $P\in L(\mathbb{F}^n)$. Then $y:=z\circ P$ is another basis for $V$.
	\item As a consequence, we have the following results:
	\item Given 2 bases $y,z$ and $P:=z^{-1}\circ y$, we have $M(T, y)=M(P^{-1},e)M(T, z)M(P, e)$.
	\item Given a basis $z$, isomorphism $P\in L(\mathbb{F}^n)$ and $y:=z\circ P$, we have $M(T, y)=M(P^{-1},e)M(T, z)M(P, e)$
\end{itemize}
My point here is that we can either go from $z,P$ to $y$, or from $z,y$ to $P$.


\textbf{Proof} The proof is by induction on dimension of $V$, $n$. Suppose $P(n)$ true. That is, for any $n$-dimensional vector space $V$, $T\in L(V)$, there exists a choice of basis $y$ such that $M(T,y)$ is upper triangular.

Let $V$ be a $n+1$-dimensional complex vector space. Let $T\in L(V)$. We define the basis $z$ as follows. We know that $T$ must have at least $1$ eigenvalue, and hence some eigenvector. So we let $z_1$ be an eigenvector of $T$, and extend $z_1$ arbitrarily to a basis $z$ for $V$. 

Consider $M(T,z)$. The first column of $M(T,z)$ is $\begin{pmatrix}
	\lambda_1\\
	0\\
	\dots \\
	0
\end{pmatrix}$ by our choice of $z_1$. Hence $M(T,z)$ has the form
\begin{align*}
	\begin{pmatrix}
		\lambda_1 & \mathbf{r} \\
		\mathbf{0} & Q
	\end{pmatrix}
\end{align*}
where $\mathbf{r}\in \mathbb{C}^{1,n}, \mathbf{0}\in \mathbb{C}^{n,1}, Q\in \mathbb{C}^{n,n}$

Note that $Q$ is an $n*n$ matrix, and hence can be viewed as a linear operator on $\mathbb{C}^n$. So by induction hypothesis, there exists a matrix $M(P_1,e)$ such that $P_1^{-1}QP_1$ is upper triangular.\\
Now define $M(P,e)=\begin{pmatrix}
	1 & \mathbf{0}\\
	\mathbf{0} & M(P_1, e)
\end{pmatrix} 
$ and we have the desired $P$.

See block matrix multiplication for more details.


\textbf{Theorem} The eigenvalues of a linear map $T$ are precisely the diagonal entries of an upper triangular matrix $A$ of $T$. 

\textbf{Proof}
First, we note that for any scalar $\lambda$ and any basis $z$, 
\begin{align*}
	M(T-\lambda I, z)&=M(T,z)-\lambda M(I,z)\\
	&=A-\lambda I.
\end{align*}
Note that $I$ in the first line is the identity map and $I$ in the second line is the identity matrix. 
 
Hence, $\lambda$ is a diagonal entry of $A$ \\
$\iff$ $A-\lambda I$ is a non-invertible/singular matrix \\
$\iff$ $T-\lambda I$ is a non-invertible/singular linear operator\\
$\iff$ $\lambda$ is an eigenvalue of $T$.


\subsection{Jordan Canonical Form}
See lecture notes for definition of Jordan block and Jordan form.

As with any upper triangular matrix of a linear operator $T$, the diagonal entries of the Jordan canonical form of $T$ are precisely the eigenvalues of $T$.\\
\textbf{Definition} The multiplicity of an eigenvalue $\lambda$ is the sum of the sizes of Jordan blocks corresponding to that eigenvalue.\\
\textbf{Remark} It is incorrect to say the multiplicity of an eigenvalue $\lambda$ is simply the number of occurrences of $\lambda$ along any upper triangular matrix. This is because the Jordan canonical form has certain desired characteristics that makes this definition of multiplicity useful in the Cayley-Hamilton Theorem.\\
\textbf{Remark 2} It turns out that the multiplicity of an eigenvalue $\lambda$ can be the number of occurrences of $\lambda$ along any upper triangular matrix(same as in the Jordan matrix form). The proof of this, however, is non-trivial and requires the notion of generalised eigenvalues. 

\textbf{Definition} Given that $\lambda_i, 1\leq i\leq k$ are precisely the eigenvalues of a linear operator $T$, and that $m_i$ is the multiplicity of eigenvalue $\lambda_i$, the the characteristic polynomial of $T$ is 
\begin{align*}
	\chi_T(x)=\prod_{i=1}^k (x-\lambda_i)^{m_i}
\end{align*}
\textbf{Remark} A reminder that eigenvalues and eigenvectors are properties of a linear operator, independent of choice of basis. Hence the characteristic polynomial is also a property of an operator, independent of basis.


\subsubsection{Cayley-Hamilton Theorem}
\begin{align*}
	\chi_T(T)=0_{L(V)}
\end{align*}

To prove this, let $y$ be a Jordan basis of $T$, such that $M(T, y)$ is in Jordan canonical form. We can show via block matrix multiplication that $\chi_T(M(T,y))=0_{\mathbb{C}^{n,n}}$.

Since $M(\chi_T(T), y) = \chi_T(M(T, y)) = 0_{\mathbb{C}^{n,n}}$, we have $\chi_T(T)=0_{L(V)}$.

\textbf{Proposition} Let $z$ be any basis of $V$. Then 
\begin{align*}
	\chi_T(M(T,z))=0_{\mathbb{C}^{n,n}}
\end{align*}

\textbf{Proof} Since $\chi_T(T)=0_{L(V)}$, i.e. $\chi_T(T)$ is the zero map, clearly any matrix of $\chi_T(T)$ is the zero matrix regardless of basis. (since $M(\chi_T(T), z) = \chi_T(M(T, z))$) Hence, we have the desired conclusion.

Additionally, let $A$ be any matrix that is similar to $J=M(T, y)$. Then there exists invertible matrix $P$ such that $A=P^{-1}JP$. Let $Q\in L(\mathbb{C}^n)$, $P=M(Q, e)$. Define $z=y\circ Q$. Then $Q$ is a change of basis isomorphism. By our proposition above,
\begin{align*}
	\chi_T(A) = \chi_T(M(T,z)) = 0_{\mathbb{C}^{n,n}}
\end{align*}

An application of the Cayley-Hamilton Theorem is that when taking powers of linear operators, higher powers can be expressed in terms of lower powers.

\textbf{Corollary} Suppose the characteristic polynomial of linear operator $T$ has degree $k$. Then all powers of $T^j, j\geq k$ can be expressed as a linear combination of $I, T, \dots, T^{k-1}$.\\
Similarly, if $T$ is invertible, then $T^{-1}$ can be expressed as a linear combination of $I, T, \dots, T^{k-1}$.




\paragraph{Stuff: To be integrated into the inner products section}

Given linear operator $T\in L(V)$, where $V$ is a complex vsp. Let $z$ be \textit{any} basis of $V$. Let $y$ be an orthonormal eigen-basis.

Let $M(P, e, e)$ be the matrix $M(I, y, z)$.
Then $y_j = \sum_{i}P_{i,j}z_i$.

Prof's Claim: The columns of $M(P)$, $c_i=\sum_{i}P_{ij}e_i$ are orthonormal.

It suffices to show $dot(c_i, c_j)=\delta_{ij}$

$M(P)=\begin{pmatrix}
c_1 \dots c_n
\end{pmatrix}
$
\begin{align*}
dot(c_i, c_j) &= dot(\sum_kP_{ki}e_k,\sum_lP_{lj}e_l)\\
&=\sum_{k,l}P_{ki}P_{lj}dot(e_k,e_l)\\
&=
\sum_{k}P_{ki}P_{kj}
\end{align*}

\begin{align*}
\delta_{ij} &= g(y_i,y_j)\\&=g(zc_i, zc_j) \\&= \sum_{k,l}P_{ki}P_{lj}g(z_k,z_l)
\end{align*}

Counterexample:

\begin{align*}
\begin{pmatrix}
1 & 1\\
0 & 1 
\end{pmatrix}
\begin{pmatrix}
1 & -1\\
0 & 1 
\end{pmatrix}
\end{align*}

Hence, this is a slight inaccuracy in the lecture notes.

We need $z$ to be orthonormal for this proof to follow through.



\section{Inner Product Spaces}
\subsection{Bilinear forms}
Given a vector space $V$ of dimension $n$, and $B(V)$, the set of all bilinear forms over $V$ (which is also a vector space), the basis of $B(V)$ is given by
\begin{align*}
	\{\zeta_i\otimes \zeta_j: 1\leq i,j\leq n\}
\end{align*} 
where $\zeta$ is a dual basis of $V$.

Denote the matrix of a bilinear form $g$ with respect to basis $z$ of $V$ as $M(g, z)$ (the $z$ here refers to both inputs to $g$).
For each $1\leq i,j\leq n, M(g,z)_{i,j}=g(z_i,z_j)$.

\subsubsection{Change of basis}
Let $z,y$ be 2 bases. Let $P=z^{-1}\circ y$ be the change of basis from $z$ to $y$.

Recall from the previous chapter, we can express $z_i,y_i,\zeta_i,\gamma_i$ in terms of each other as follows:
TODO


\subsection{Real inner products}
A real inner product $g$ is a bilinear form that has the following properties:
\begin{enumerate}
	\item Positivity: $\forall v\in V, g(v,v)\geq 0$
	\item Definiteness: $g(v,v)=0\iff v=0$
	\item Symmetry: $\forall u,v\in V, g(u,v)=g(v,u)$
\end{enumerate}
\subsubsection{Matrix of an inner product}
Let $M(g,z)$ be the matrix of real inner product $g$ with respect to basis $z$ of $V$. Denote $g_{i,j}=M(g,z)_{i,j}$.\\
Let $u,v\in V$. Then $u=\sum_i a_iz_i,\, v=\sum_i b_iz_i$ for some scalars $a_i,b_i\in \mathbb{R}$.\\
The matrix of a real inner product has these properties:
\begin{enumerate}
	\item Positivity: $g(u,u)=\sum_{i,j} a_ia_j g_{i,j}\geq 0$
	\item Definiteness: $\sum_{i,j} a_ia_j g_{i,j}=0 \iff \forall 1\leq i\leq n, a_i=0$
	\item Symmetry: $g_{i,j}=g_{j,i}$
\end{enumerate}
\textbf{Remark} $g_{i,j}=g_{j,i}$ is a sufficient and necessary condition for $g$ to be symmetric.

\textbf{Proposition} Computing inner products by matrix multiplication\\
Use the same definitions of $g_{i,j},u,v$ as above. Then the 1*1 matrix of $g(u,v)$ is given by
\begin{align*}
	[g(u,v)] = M(u,z)^TM(g,z)M(v,z)
\end{align*}
\textbf{Proof} Observe that 
\begin{align*}
	[g(u,v)]_{1,1} = \sum_{i,j}a_ib_j g_{i,j} = \sum_{i,j}a_ig_{i,j}b_j = \sum_{i,j}(M(u,z)^T)_{1,i} M(g,z)_{i,j} M(v,z)_{j,1}
\end{align*}
and the result follows.

\textbf{Proposition} Matrix of an inner product cannot have non-positive eigenvalues.

\textbf{Proof} Let $A$ be a matrix of real inner product $g$. Suppose $A$ has a non-positive eigenvalue $\lambda \leq 0$. Then there exists eigenvector $v\in \mathbb{R}^n, v\neq 0$, such that $Av=\lambda v$. Note that $v$ corresponds to some vector $u\in V$, i.e. $M(u, z)=v$.\\
Hence, $g(u,u)=v^TAv=v^T\lambda v=\lambda v^Tv = \lambda |v|^2 \leq 0$. (Note that the norm here is the Euclidean norm since $v\in \mathbb{R}^n$).\\
Then $g$ is not positive definite. Hence $g$ cannot be an inner product.

\subsection{Orthogonality}
\textbf{Definition} $z$ is an orthonormal basis of real inner product space $(V,g)$ if for each $1\leq i,j\leq n$, $g(z_i,z_j)=\delta_{i,j}$.\\
Hence $M(g,z)$ is the identity matrix.

An orthonormal basis can also be defined in the "mapping" way.\\
\textbf{Definition} $z$ is an orthonormal basis of real inner product space $(V,g)$ if $\forall a,b\in \mathbb{R}^n$,
\begin{align*}
	g(za, zb) = dot(a,b)
\end{align*}
(Note: dot here is the dot product)
We can prove that the 2 definitions are indeed equivalent.

\textbf{Proposition} Pythagorean Theorem\\
Let $(y_i)_{1\leq i\leq k}$ be an orthonormal list of vectors. Then 
\begin{align*}
	\left\lvert \sum_{1\leq i\leq k}a_iy_i\right\rvert^2 = \sum_{1\leq i\leq k}a_i^2
\end{align*}

\textbf{Corollary} Orthonormal list of vectors are linearly independent.

\subsection{Riesz Representation Theorem (Real)}
Given inner product space $(V,g)$, define the linear map
\begin{align*}
	\Gamma: &V\rightarrow \hat{V}\\
	&u\mapsto \Gamma_u
\end{align*}
where $\forall v\in V, \Gamma_u(v) = g(u,v)$.

Since $dim(V)=dim(\hat{V})$, to show that $\Gamma$ is an isomorphism, it suffices to show that this is a injection. And indeed, $\Gamma(u)=0_{\hat{V}}\implies \Gamma(u)(u)=g(u,u)=0\implies u=0$. Since kernel of $\Gamma$ is trivial, it is injective.

In particular, there being a bijection between $V$ and $\hat{V}$ means that 
\begin{align*}
\forall \alpha \in \hat{V}, \exists! u\in V, \Gamma(u) = \alpha
\end{align*}

\textbf{Proposition} Cauchy Schwarz inequality\\
\begin{align*}
	|g(u,v)| \leq |u||v|
\end{align*}
The equality cases occurs iff $u=cv, c\in \mathbb{R}$.

\textbf{Proof}
Write $u = \frac{g(u,v)}{|v|^2}v + (v-\frac{g(u,v)}{|v|^2}v)$.\\
Denote $p=\frac{g(u,v)}{|v|^2}v$, $n=u-p$.\\
Then
\begin{align*}
	|u|^2 &= |p|^2 + |n|^2 &&\text{by orthogonality}\\
	&\geq |p|^2\\
	&= \left\lvert \frac{g(u,v)}{|v|^2}v \right\rvert^2 \\
	&= \frac{g(u,v)^2}{|v|^4} |v|^2\\
	&= \frac{g(u,v)^2}{|v|^2}
\end{align*}
Multiplying $|v|^2$ across gives the desired inequality.\\
The equality case occurs iff
\begin{align*}
	|n|^2=0 \implies n=0\implies u=p \implies u = \frac{g(u,v)}{|v|^2}v
\end{align*}
Setting $c=\frac{g(u,v)}{|v|^2}$, we conclude the proof.

\textbf{Proposition} Triangle Inequality
\begin{align*}
	|u+v| \leq |u| + |v|
\end{align*}
If $v\neq 0$, the equality case occurs iff $u=cv, c\in \mathbb{R}^+_0$

\textbf{Proof}
\begin{align*}
	|u+v|^2 &= |u|^2 + g(u,v) + g(v,u) + |v|^2\\
	&= |u|^2 + 2g(u,v) + |v|^2\\
	&\leq |u|^2 + 2|u||v| + |v|^2\\
	&\leq (|u| + |v|)^2
\end{align*}
The equality cases occurs iff $g(u,v)=|u||v|$. Cauchy Schwarz equality case tells us that $u=cv$ for some $c\in \mathbb{R}$. Substituting in gives $c|v|^2=|c||v|^2\implies c=|c|$ assuming $v$ non-zero. Hence, $c$ must be non-negative.

\subsubsection{Component form of Riesz Representation Theorem}
Given a basis $z$ of $V$. Let $\alpha=\sum_i p_i\zeta_i$.
Then the Riesz theorem asserts that there exists a unique $u=\sum_i a_iz_i\in V$ such that for all $v=\sum_i b_iz_i \in V$,
\begin{align*}
	\alpha(v) &= g(u,v)\\
	\sum_{i,j}p_ib_j\zeta_i(z_j) &= \sum_{i,j}a_ib_jg_{i,j}\\
	\sum_j p_jb_j &= \sum_j \sum_i a_ig_{i,j}b_j
\end{align*}
Since the scalars $b_i$ are arbitrary, by setting $b_j=1, b_k=0\forall k\neq j$, we obtain the equalities
\begin{align*}
	p_j = \sum_ia_ig_{i,j}
\end{align*}
If we view $\begin{pmatrix}
p_1 & p_2 & \dots & p_n
\end{pmatrix}$,$\begin{pmatrix}
a_1 & a_2 & \dots & a_n
\end{pmatrix}$  as a row vectors,
our equality becomes 
\begin{align*}
	p_{1,j} = \sum_ia_{1,i}g_{i,j}
\end{align*}
It is immediately apparent the the row vector $\begin{pmatrix}
p_1 & p_2 & \dots & p_n
\end{pmatrix}$ is the product of $\begin{pmatrix}
a_1 & a_2 & \dots & a_n
\end{pmatrix}$ and the matrix $M(g,z)$.

As proven previously, the matrix of an inner product cannot have non-positive eigenvalues. In particular, 0 is not an eigenvalue. And this says that the matrix $M(g,z)$ is invertible.

Hence, 
\begin{align*}
\begin{pmatrix}
a_1 & a_2 & \dots & a_n
\end{pmatrix} = \begin{pmatrix}
p_1 & p_2 & \dots & p_n
\end{pmatrix}(M(g,z))^{-1}
\end{align*}

Hence,
\begin{align*}
	a_j=a_{1,j} = \sum_ip_{1,i}(M(g,z)^{-1})_{i,j} = \sum_ip_i(M(g,z)^{-1})_{i,j} = \sum_i p_i g^{-1}_{i,j}
\end{align*}
where $g^{-1}:=M(g,z)^{-1}$

Since $g$ is symmetric, $M(g,z)^{-1}$ is a symmetric matrix.
Hence, 
\begin{align*}
	a_j=\sum_i g^{-1}_{j,i}p_i 
\end{align*}
Here, we view $\begin{pmatrix}
a_1\\ a_2\\ \dots \\ a_n
\end{pmatrix}$ and $\begin{pmatrix}
p_1\\ p_2\\ \dots \\ p_n
\end{pmatrix}$ as column vectors.

\subsection{Extension to complex vector spaces}
Define sesquilinear forms as in lecture notes. (Linear in first component, conjugate linear in second component)

Changes to the following results:
\paragraph{Cauchy Schwarz inequality} It suffices to adjust one line in the proof
\begin{align*}
	&\left\lvert \frac{g(u,v)}{|v|^2}v \right\rvert^2\\
	&= \frac{|g(u,v)|^2}{|v|^4} |v|^2\\
	&= \frac{|g(u,v)|^2}{|v|^2}
\end{align*}

\paragraph{Triangle inequality} Again, we just need to adjust one line in the proof
\begin{align*}
	g(u,v) + g(v,u) = g(u,v) + \overline{g(u,v)} = 2Re(g(u,v))\leq 2|g(u,v)|\leq 2|u||v|
\end{align*}
Assuming $v\neq 0$, the equality case also occurs then $u=cv, c\in \mathbb{R}^+_0$. The proof is as follows:

First we note that $|g(u,v)|=|u||v|\iff u = cv, c\in \mathbb{C}$

Then equality occurs iff 
\begin{align*}
	Re(g(cv,v) = |c||v|^2\implies Re(c|v|^2)=|c||v|^2\implies Re(c)=|c|
\end{align*}
The last equality says that $c$ must be non-negative.

\paragraph{Riesz Representation Theorem}
Since the complex inner product $g$ is a sesquilinear form, we will need to adjust our definition of $\Gamma$.
Define 
\begin{align*}
	\Gamma: &V\rightarrow \hat{V}\\
	&u\mapsto \Gamma_u
\end{align*}
where $\forall v\in V, \Gamma_u(v) = g(v, u)$

Note that $\Gamma$ is no longer a linear map, since $\Gamma(\lambda u) = \overline{\lambda}\Gamma(u)$.
However, we can still show that $\Gamma$ is a conjugate isomorphism.\\
In particular, $\Gamma$ is a bijection between $V$ and $\hat{V}$.

\textbf{Component form}
\begin{align*}
	&\alpha(v) = g(v,u)\\
	&\sum_{i,j}p_ib_j\zeta_i(z_j)=\sum_{j,i}b_j\overline{a_i}g(z_j,z_i)=\sum_{j,i}b_j\overline{a_i}g_{j,i}\\
	&\sum_j p_jb_j = \sum_j\sum_i \overline{a_i}g_{j,i} b_j\\
	&p_j = \sum_i g_{j,i}\overline{a_i}
\end{align*}
Writing the last eqn by interpreting $p_j$, $a_i$ as column vectors, we get,
\begin{align*}
	p_{j,1} = \sum_i g_{j,i}\overline{a_{i,1}}
\end{align*}
Hence multiplying the inverse of $M(g,z)$ over gives
\begin{align*}
	\overline{a_j}=\sum_i g^{-1}_{j,i}p_i
\end{align*}
Here we denote $M(g,z)^{-1}$ as $g^{-1}$. To obtain $a_j$, simply apply conjugation on both sides of the equality.
\paragraph{Schur's Theorem}
Given a linear operator $T$ on a finite dimensional complex inner product space, there exists an orthonormal basis $z$ such that $M(T,z)$ is upper-triangular.

\textbf{Proof} First choose a basis $y$ of $V$ such that $M(T,y)$ is upper triangular. Then apply Gram-Schmidt to $y$ to obtain orthonormal basis $z$. Then the change of basis matrix $M(P)$ from $y$ to $z$ is upper-triangular. \\
Also, the inverse of an upper-triangular matrix $M(P)$ is upper-triangular (Proven by the corollary to Cayley-Hamilton theorem). Since the product of upper triangular matrices is upper triangular, $M(T,z) = M(P)^{-1}M(T,y)M(P)$ is also upper triangular.

\subsection{Spectral Theorem}
\textbf{Lemma} Given inner product space $(V,g)$, there is a bijection between the set of sesquilinear forms $S(V)$ and the set of linear operators $L(V)$ defined by the relation
\begin{align*}
	\forall u,v \in V,\, \tau(u,v) = g(u,Tv) 
\end{align*}

We now consider the above relation in component form. Let $z$ be a basis of $V$. Letting $u,v$ be the basis vectors of $z$, we have
\begin{align*}
	\tau_{i,j} = \tau(z_i,z_j) = g(z_i,Tz_j) = g(z_i,\sum_k T_{k,j}z_k) = \sum_k \overline{T_{k,j}}g(z_i, z_k)=\sum_k \overline{T_{k,j}}g_{i,k}
\end{align*}
Hence, 
\begin{align*}
	\tau_{i,j} = \sum_k g_{i,k}\overline{T_{k,j}}
\end{align*}


\textbf{Definition} A sesquilinear form $\tau$ is hermitian if $\forall u,v\in V,\, \tau(u,v)=\overline{\tau(v,u)}$.\\
Hermetian is essentially the same as conjugate symmetric.

\textbf{Definition} A linear operator $T$ over $(V,g)$ is Hermitian iff its corresponding sesquilinear form $\tau$ is Hermitian.\\
Note that we can only talk about Hermitian operators when $V$ is equipped with an inner product.


\textbf{Lemma} If $T$ is Hermitian, then $\forall u,v \in V, g(u,Tv)=g(Tu,v)$.

\textbf{Corollary} If $T$ is Hermitian, $T$ has only real eigenvalues.

\textbf{Proof} Let $\lambda \in \mathbb{C}$ be an eigenvalue of $T$. Let $v$ be an eigenvector of $T$ associated with $\lambda$. Then, setting $u=v$,
\begin{align*}
	g(v,Tv) = g(Tv,v) \implies \lambda |v|^2=\overline{\lambda} |v|^2 \implies \lambda=\overline{\lambda}
\end{align*}
Hence, $\lambda \in \mathbb{R}$.

\textbf{Lemma} With respect to an orthonormal basis, the matrix of $T$ is the conjugate of the matrix of $\tau$.

\textbf{Proof} Recall that 
\begin{align*}
	\tau_{i,j} = \sum_k g_{i,k}\overline{T_{k,j}}
\end{align*}
Since we are given that $z$ is orthonormal, $g_{i,k}=\delta_{i,k}$ and 
\begin{align*}
	\tau_{i,j} = \overline{T_{i,j}}
\end{align*}

\textbf{Corollary} If $T$ is Hermitian, that with respect to an orthonormal basis $z$, the matrix of $T$ is also Hermitian.

\textbf{Proof} The result follows by:
\begin{align*}
	T_{i,j} &= \overline{\tau_{i,j}} \\
	&= \tau_{j,i}\quad \text{since }\tau \text{ hermitian}\\
	&= \overline{T_{j,i}}
\end{align*}

\paragraph{Spectral Theorem (Complex)} Given a Hermitian linear operator $T$, there exists an orthonormal basis $z$ such that $M(T,z)$ is diagonal.

\textbf{Remark} Since $M(T,z)$ is diagonal, $z$ is made up of eigenvectors.

\textbf{Proposition} Suppose $y$ is an orthonormal basis. Let $T$ be Hermitian. Let $z$ be the orthonormal basis where $M(T,z)$ is diagonal. Then the change of basis matrix from $y$ to $z$ is unitary.

\textbf{Remark} The condition that $M(T,z)$ is diagonal is not necessary. In fact, given any two orthonormal bases $y,z$, the change of basis matrix between these is orthogonal.

\paragraph{Matrix form of Spectral Theorem (Complex)} Given a Hermitian matrix $A$, the exists a unitary matrix $U$ such that $\overline{U^T}AU$ is diagonal.

\textbf{Proof} $A$ can be interpreted as the matrix of a linear operator $T$ in $L(\mathbb{C}^n)$ with respect to the standard basis $e$, which is orthonormal. By the Spectral Theorem, there exists an orthonormal basis $z$ made up of eigenvectors of $T$. Then the matrix $U$ of the change of basis transformation from $e$ to $z$ is unitary.
Hence, $\overline{U^T}AU=\overline{U^T}M(T,e)U = M(T,z)$, where $M(T,z)$ is diagonal.

\textbf{Remark} Notice that this is slightly different from the spectral theorem. In the operator form of the spectral theorem, we did not claim that the matrix of the change of basis transformation is unitary. That is because we did not necessarily start out with an orthonormal basis $y$.

In MA2101, the real forms of the spectral theorem (and its matrix variation) are stated without proof.

\paragraph{Spectral Theorem (Real)} Given a symmetric linear operator $T$ over real inner product space $V$, there exists an orthonormal basis $z$ such that $M(T,z)$ is diagonal.

\paragraph{Matrix form of Spectral Theorem (Real)} Given a symmetric matrix $A$, the exists an orthogonal matrix $O$ such that $O^TAO$ is diagonal.


\subsection{Determinants}
Define multilinear forms as in lecture notes.
Define n-forms as in lecture notes. In short, $n$-forms are multilinear forms that are antisymmetric with respect to any pair of components.

\paragraph{Properties of 2-forms}
Let $f$ be a 2-form. Then by definition, $f(u,v)=-f(v,u)$.\\
Hence, $f(u,u) = -f(u,u)\implies f(u,u)=0$.

Define the wedge product 
\begin{align*}
\zeta_1\wedge \dots, \wedge \zeta_k := \frac{1}{k!}(\zeta_1\otimes \zeta_2\otimes \dots, \otimes \zeta_k - \zeta_2\otimes \zeta_1\otimes \dots, \otimes \zeta_k + \dots)
\end{align*}

For 2 dual vectors, 
\begin{align*}
\zeta_1\wedge \zeta_2 := \frac{1}{2!}(\zeta_1\otimes \zeta_2 - \zeta_2 \otimes \zeta_1)
\end{align*}
In other words, we enumerate all permutations of $\zeta_1, \dots, \zeta_k$ and each "swap" we make, we multiply a factor of $-1$.

Let $V$ be a vector space of dimension $n$, and let $z$ be a basis of $V$, $\zeta$ its dual basis.

We will prove that $\{\zeta_i \wedge \zeta_j : 1\leq i < j \leq n\}$ is a basis for the set of 2-forms on $V$.

Let $f$ be a 2-form, then $f$ is a bilinear form, and hence 
\begin{align*}
f &= \sum_{1\leq i,j\leq n}f(z_i, z_j)\zeta_i\otimes \zeta_j\\
&=\sum_{1\leq i\leq n}f(z_i, z_i)\zeta_i\otimes \zeta_i
+ \sum_{1\leq i < j\leq n}f(z_i, z_j)\zeta_i\otimes \zeta_j
+ \sum_{1\leq j < i\leq n}f(z_i, z_j)\zeta_i\otimes \zeta_j\\
&= 0 + \sum_{1\leq i < j\leq n}f(z_i, z_j)\zeta_i\otimes \zeta_j
+ \sum_{1\leq j < i\leq n}-f(z_j, z_i)\zeta_i\otimes \zeta_j\\
&=\sum_{1\leq i < j\leq n}f(z_i, z_j)\zeta_i\otimes \zeta_j
+ \sum_{1\leq i < j\leq n}-f(z_i, z_j)\zeta_j\otimes \zeta_i\\
&= \sum_{1\leq i < j\leq n}(2!)f(z_i, z_j)\zeta_i\wedge \zeta_j
\end{align*}
which completes the proof.

The dimension of the set of 2-forms is $nC2$. In general, the dimension of the set of $m$-forms is $nCm$. The dimension of the set of $n$-forms is $nCn=1$.

Let $T$ be a linear operator on $V$. The transpose of $T$ with respect to an $m$-form $\Omega$ is defined as follows:\\
$\hat{T}\Omega(v_1, \dots, v_m) = \Omega(Tv_1, \dots, Tv_m)$.\\
Hence, $\hat{T}\Omega$ is another $m$-form.

If $m=n$, then $\hat{T}\Omega \in span(\Omega)$. Hence, we can associate a number $\Delta_\Omega(T)$, such that 
\begin{align*}
	\hat{T}\Omega = \Delta_\Omega(T)\Omega
\end{align*}

Let $z$ be a Jordan basis of $V$ with respect to $T$, and $z_i, 1\leq i\leq n$ are the Jordan basis vectors. Then letting $\hat{T}\Omega$ act on $z$, we have
\begin{align*}
	\hat{T}\Omega(z_1, \dots, z_n)
	&= \Omega(Tz_1, \dots, Tz_n) \\
	&= \prod_{i=1}^n T_{i,i} \Omega(z_1, \dots z_n)
\end{align*}
Hence, $\Delta_\Omega(T) = \prod_iT_{i,i}$, the product of the diagonal entries of the matrix of $T$.\\
Now, notice that this holds for any choice of $\Omega$. Hence we can ditch the $\Omega$ subscript, and write
\begin{align*}
	\hat{T}\Omega = \Delta(T)\Omega
\end{align*}
In other words, this is a property of $T$, independent of choice of $\Omega$.

\textbf{Remark} If we are to view $J=M(T,z)$, the matrix of $T$ w.r.t. the Jordan basis as a linear operator over $\mathbb{F}^n$, then it is clear that $\Delta(J)=\Delta(T)$, i.e. the product of the diagonal entries of $J$. This trivial observation that tells us that the determinant of a linear operator $T$ is the same as the determinant of any matrix of $T$ under any choice of basis.

This allows us to derive other properties of $\Delta$ by choosing any arbitrary $\Omega$. Some properties derived in lecture are:
\begin{itemize}
	\item If $S,T$ are linear operators, then $\Delta(ST) = \Delta(S)\Delta(T)$
	\item An immediate consequence is that $\Delta(P)\Delta(P^{-1})=1$
	\item Let $A, B$ be similar matrices. Then $B=P^{-1}AP$ for some invertible matrix $P$. We can view $B, A, P, P^{-1}$ as linear operators in $\mathbb{F}^n$, hence $\Delta(B)=
	\Delta(P^{-1})\Delta(A)\Delta(P)=\Delta(A)$
\end{itemize}

We now refer to $\Delta$ as $\det$.

\subsubsection{More properties of det}
By default, we will let $A$ denote a square matrix, and $T$ denote a linear operator over n-dimensional vector space $V$.
\paragraph{Transpose of a matrix preserves determinants}
We first prove that a Jordan matrix has the same determinant as its transpose.
If $J$ is a Jordan matrix, then it's determinant is just the product of the diagonal entries, as we have proven above in $\Delta_\Omega(T) = \prod_iT_{i,i}$. Since $J^T$ is a lower triangular matrix, the antisymmetric property of an $n$-form means that its determinant is also the product of the diagonal entries. Hence $\det(J)=\det(J^T)$.

If $A$ is a complex square matrix, then it is similar to a Jordan matrix $J$. Also, $A^T$ is similar to $J^T$.\\
To see this, write
\begin{align*}
	P^{-1}AP=J\\
	P^TA^T(P^{-1})^T=J^T\\
	P^TA^T(P^T)^{-1}=J^T
\end{align*}
Hence
\begin{align*}
	\det(A^T) = \det(J^T) = \det(J) = \det(A)
\end{align*}


\paragraph{Conjugate of a matrix conjugates determinant}
As with the previous section, consider Jordan matrix $J$ first.
\begin{align*}
	\det(\overline{J})
	&=\prod_i\overline{J}_{i,i}\\
	&=\prod_i\overline{J_{i,i}}\\
	&=\overline{\prod_iJ_{i,i}}\\
	&=\overline{\det(J)}
\end{align*}

Then for a complex matrix $A$, it is similar to some Jordan matrix $J$. Also, $\overline{A}$ is similar to $\overline{J}$.
To see this, write
\begin{align*}
	&P^{-1}AP=J\\
	&\overline{P^{-1}AP}=\overline{J}\\
	&\overline{P^{-1}}\,\overline{A}\,\overline{P}=\overline{J}\\
	&(\overline{P})^{-1}\, \overline{A}\, \overline{P}=\overline{J}
\end{align*}
Hence,
\begin{align*}
	\det(\overline{A}) = \det(\overline{J}) = \overline{\det(J)} = \overline{\det(A)}
\end{align*}

\textbf{Remark} In the above, we also used some minor propositions along the way. 
\begin{itemize}
	\item Transpose and inverse can be interchanged
	\item Conjugate and inverse can be interchanged
	\item Transpose and conjugate can be interchanged (this is not mentioned above, but this is easy to see)
\end{itemize}

\section{Miscellaneous Lemmas}
\paragraph{Necessary condition for a matrix to be the matrix of an inner product}

Suppose a real square matrix $A$ has non-positive ($0$ or negative) eigenvalues. Then $A$ cannot be the matrix of an inner product.

\textbf{Proof} Suppose $-\lambda$ (where $\lambda \geq 0$ is an eigenvalue of $A$. Then, there exists a non-zero vector(column matrix) $v$ such that $Av=-\lambda v$.\\
Hence, $v^TAv = -\lambda v^Tv = -\lambda dot(v,v) \leq 0$.\\
Suppose $v=\sum_i a_ie_i$. Also suppose that $z=\{z_1,\dots z_n\}$ is a basis of inner product space $V$.\\
Then set $u=\sum_i a_iz_i$, and we have $g(u,u)\leq 0$. Note that $u\neq 0_V$ since not all coefficients $a_i$ are zero.\\
This says that $g$ is not positive definite. Hence $g$ is not an inner product.

\paragraph{Linear isometries in $\mathbb{R}^n$ are precisely orthogonal matrices}
(Technically an isometry is a linear map in $L(\mathbb{R}^n)$ and is not quite the same as $\mathbb{R}^{n,n}$, but we'll be a bit sloppy here.)

\paragraph{Parallelogram law} Let $g$ be a real inner product, and $|\cdot|$ is a norm induced by $g$. Then $g(u,v) = \frac{1}{4}(|u+v|^2-|u-v|^2)$.\\
\textbf{Remark} Notice that this allows us to express the real inner product in terms of the induced norm.

\end{document}




