\documentclass{article}
\usepackage[english]{babel}
\usepackage[utf8]{inputenc}
\usepackage{amsmath,amssymb}
\usepackage{parskip}
\usepackage{graphicx}
\usepackage{hyperref}
\newcommand\tab[1][1cm]{\hspace*{#1}}

% Margins
\usepackage[top=2.5cm, left=3cm, right=3cm, bottom=4.0cm]{geometry}


\title{MA2101 (Linear Algebra 2)}
\author{Jia Cheng}
\date{January 2021}

\begin{document}

\maketitle

\section{Definitions and Formula}


\section{General}
Reference Book: Axler's Linear Algebra Done Right

\subsection{Well-defined functions}
I am inspired to discuss this by the section on quotient spaces.\\
For a vector space $V$, subspace $U$ of $V$, define $V/U=\{v+U : v\in V\}$.\\
Define vector addition on $V/U$ by $(v + U) + (w + U) = ((v+w) + U)$.
Define scalar multiplication on $V/U$ by $\lambda(v+U)=((\lambda v)+U)$.

One of the steps in proving that our newly defined $+$ is a binary operation on $V/U$, is to show that it is well-defined. That is, for each distinct element in the domain ($V/U$), there exists a unique image in the codomain.

For sets like integers, the representation of any element in the domain is unique, so we usually don't have to worry about well-definition of functions. However, in the case of quotient spaces, $v-v' \in U$ implies $(v+U)=(v'+U)$, even though we can have $v\neq w$. It is because of these multiple representations that we have to worry about "well-defined-ness". We need to make sure when $(v+U)=(v'+U),(w+U)=(w'+U)$, we have $(v+U)+(w+U)=(v'+U)+(w'+U)$.

\subsection{Matrices vs Linear Maps}
Other than determinants and the section on eigenvectors, MA1101R (Linear Algebra 1) is sufficiently rigorous.
The key difference between MA1101R and Axler's Linear Algebra is the approach taken to building up the theory.

An advantage of thinking in terms of linear maps rather than matrices is that linear maps are independent of our choice of bases. The same matrix can do very different things if we decide to change up the bases, but the same linear map remains invariant.

This is especially useful in abstract vector spaces. A generic vector space $V$ does not have a canonical basis (unlike the standard basis in $\mathbb{R}^n$ or polynomials $P(\mathbb{R})$. We can play around with linear maps without having the need to state a basis. (whereas we do if we want to write things in matrix form, or in simultaneous equation form)

An example of the usefulness of this is seen here: \url{https://math.stackexchange.com/questions/3749/why-do-we-care-about-dual-spaces}.
Linear functionals (linear maps that have the field as the codomain) allow the definition of a hyperplane without the need for a basis.

To quote Matt E, the author of the post:\\
So this gives a reasonable justification for introducing the elements of the dual space to $V$; they generalize the notion of linear equation in several variables from the case of $\mathbb{R}^n$ to the case of an arbitrary vector space.


\section{Matrices, A revision}
\subsection{Change of basis}
For simplicity, we consider linear operators in $\mathbb{R}^2$.

The following notations $M(T, B1, B2), M(v, B)$ (respectively matrix representation of a linear transformation $T$, matrix representation of a vector) are found in Axler's book.

Suppose we have 2 bases, the standard basis $(i, j)$ and another basis $(u,v)$. Now define the matrix $P=
\begin{bmatrix}
	M(u, (i,j)) & M(v, (i,j))
\end{bmatrix}
$
Then we will obtain the following equations.
\begin{align*}
	u=Pi\\
	v=Pj
\end{align*}
There are 2 ways to understand/interpret this equality.
\begin{enumerate}
	\item First, we "see" $P$ as purely a matrix, and we operator solely in the canonical basis of $\mathbb{R}^2$. Then this equality is simply an equality of matrices. i.e. When the standard basis vector $i$ is left multiplied by matrix $P$, we obtain the vector $u$, also with respect to the standard basis.
	\item Second, we view the equation as a change of basis. To be precise, the first equality can be written as $M(u, (i,j))=M(I, (u,v), (i,j))M(u, (u,v))$. The second equality is written similarly.
	\item Note that regardless of the viewpoint, the above 2 equalities are equalities in matrices! Clearly, $i$ and $u$ are not the same vector.
\end{enumerate}

\subsubsection{A possible area of confusion}
In this course (or at least chapter 1), Prof Brett usually represents multiplication of matrix $P$ with vector $v$ as $Pv$. Note that to be fully rigorous, we need to differentiate between an abstract vector and a matrix representation of a vector. For example, if we were to talk about a real polynomial $v$, which is a vector in $P(\mathbb{R})$, then obviously, matrix multiplication with $v$ is undefined.

So in my opinion, the more rigorous way to write things is $PM(v, B)$, where $B$ is a basis of the vector space $v$ lies in. $M(v,B)$ is that the matrix representation of $v$ w.r.t basis $B$.

Of course, in chapter 1, we do not deal with abstract vector spaces, so when Prof Brett writes $Pv$, he means that matrix $P$ is multiplied with the representation of $v$ in the standard basis of $\mathbb{R}^n$.

\subsubsection{MA1101R Recap}
Given 2 bases $B_1=(u_1,u_2),B_2=(v_1,v_2)$ of $\mathbb{R}^2$, then how do we find the change of basis matrix from $B_2$ to $B_1$, $M(I, B_1, B_2)$? (Notice the order "from $B_2$ to $B_1$". This will be explained in future.)\\
$M(I, B_1, B_2)=[M(u_1, B_2), M(u_2,B_2)]$\\
To find $M(u_i, B_2), i=1,2$, we aim to find $a,b\in \mathbb{R} s.t. av_1+bv_2=u_i$, i.e. $\begin{pmatrix}
	v_1 & v_2
\end{pmatrix}\begin{pmatrix}
	a\\
	b
\end{pmatrix}=\begin{pmatrix}
	u_i
\end{pmatrix}
$\\
Here, we did not write $M(v_1,\text{basis}),M(v_2,\text{basis}),M(u_i,\text{basis})$ within the expression, this basis is arbitrary, though ofc it has to be the same for each vector. Usually, this basis would be the canonical basis of $\mathbb{R}^2$.\\
Next, apply Gaussian Elimination to $\begin{pmatrix}
	v_1 & v_2 & | & u_1 & u_2
\end{pmatrix}$
\subsection{Markov Chains}
Given a finite set of states $E_1, \dots E_n$, let $A\in \mathbb{R}^{n,n}$ be the matrix where
\begin{align*}
	(A)_{i,j}=P(E_{i,m+1}\mid E_{j,m})
\end{align*}
where the $E_{j,m}$ denotes the event where state $E_j$ occurs on the $m$-th trial.

Some properties of the stochastic matrix:
\begin{enumerate}
	\item Sum of a column: $\sum_{1\leq i\leq n}(A)_{i,j}= \sum_{1\leq i\leq n}P(E_i\mid E_j)=1$
	\item Power of a matrix: $(A^k)_{i,j}=P(E_{i, m+k}\mid E_{j, m})$
\end{enumerate}

For instance, to use an example from lecture notes:
\begin{itemize}
	\item $P(\text{rain next day} \mid \text{rain today}) = 0.6$
	\item $P(\text{sunny next day} \mid \text{sunny today}) = 0.7$
\end{itemize}
The stochastic matrix is then 
\begin{align*}
	M=\begin{pmatrix}
		P(R_{i+1}\mid R_i) & P(R_{i+1}\mid S_i)\\
		P(S_{i+1}\mid R_i) & P(S_{i+1}\mid S_i)
	\end{pmatrix}
	=
	\begin{pmatrix}
		0.6 & 0.3\\
		0.4 & 0.7
	\end{pmatrix}
\end{align*}
Then probability theory tells us that
\begin{align*}
	P(R_{i+2}\mid R_i) &= P(R_{i+2}\mid R_{i+1}R_i)P(R_{i+1}\mid R_i)+P(R_{i+2}\mid S_{i+1}R_i)P(S_{i+1}\mid R_i)\\
	&= P(R_{i+2}\mid R_{i+1})P(R_{i+1}\mid R_i)+P(R_{i+2}\mid S_{i+1})P(S_{i+1}\mid R_i)=(M^2)_{1,1}
\end{align*}
The second equality is due to the assumption that whether it is rainy/sunny  on day $i+2$ depends only on whether it rains/is sunny on day $i+1$.

To calculate the actual probabilities:
Define $x_n=\begin{pmatrix}
	P(R_n)\\
	P(S_n)
\end{pmatrix}
,\, \forall n\in \mathbb{N}$\\
Then \begin{align*}
	M^kx_0&=\begin{pmatrix}
		P(R_{k}\mid R_0) & P(R_{k}\mid S_0)\\
		P(S_{k}\mid R_0) & P(S_{k}\mid S_0)
	\end{pmatrix}\begin{pmatrix}
		P(R_0)\\
		P(S_0)
	\end{pmatrix}\\
	&=P(R_0)\begin{pmatrix}
		P(R_{k}\mid R_0)\\
		P(S_{k}\mid R_0)
	\end{pmatrix}+
	P(S_0)\begin{pmatrix}
		P(R_{k}\mid S_0)\\
		P(S_{k}\mid S_0)
	\end{pmatrix}\\
	&=\begin{pmatrix}
		P(R_{k}\cap R_0)\\
		P(S_{k}\cap R_0)
	\end{pmatrix}+
	\begin{pmatrix}
		P(R_{k}\cap S_0)\\
		P(S_{k}\cap S_0)
	\end{pmatrix}\\
	&=x_k
\end{align*}
\subsection{Sums and products of eigenvalues}
We claim that trace $tr$ and determinant $det$ are attributes of linear operators, regardless of the choice of basis.
\begin{align*}
&tr(P^{-1}AP)=tr(APP^{-1})=tr(A)\\
&det(P^{-1}AP)=det(P^{-1})det(A)det(P)=det(P^{-1}P)det(A)=det(A)
\end{align*}
Hence, if $P$ is the change of basis matrix from eigenvector-basis to standard basis, and $A$ is the matrix of $T$ w.r.t standard basis, we then have 
\begin{align*}
	tr(D) &= tr(A)\\
	det(D) &= det(A)
\end{align*} 
where $D$ is a diagonal matrix (matrix of $T$ w.r.t eigenvector-basis). This says that $tr(A)$ is the sum of eigenvalues of linear map $T$, and $det(A)$ is the product of eigenvalues of linear map $T$.

Geometrically (in 2 or 3 dimensions), we can then understand the magnitude of the determinant of a linear map $T$ by the scaling of the area/volume of the parallelogram/parallelepiped bordered by linearly independent eigenvectors after applying linear transformation $T$ on those eigenvectors. Prof Brett calls this the "unit box".

\subsection{Useful formula}
This result can be proven by expanding the exponential into a sum and applying multivariable calculus. (Simply think of a matrix as a vector)
\begin{align*}
	\frac{d}{dt} e^{Bt}=Be^{Bt}
\end{align*}
Note: Here, $Bt=tB$ (multiplying constant matrix $B$ by scalar $t$)

Let $D=\begin{pmatrix}
	\lambda_1 & 0\\
	0 & \lambda_2
\end{pmatrix}$.
Then,
\begin{align*}
	e^{D}=\begin{pmatrix}
		e^{\lambda_1} & 0\\
		0 & e^{\lambda_2}
	\end{pmatrix}, 
	e^{Dt}=\begin{pmatrix}
		e^{\lambda_1t} & 0\\
		0 & e^{\lambda_2t}
	\end{pmatrix}
\end{align*}
where $t$ is a scalar.


\section{Vector spaces}
In the following chapter, Prof Brett diverges quite significantly from Axler's book. However, what this shows is that Mathematics truly is a "web" as Prof Han Fei says, instead of a series of linear progressions. i.e. There are many places/foundations that we can start from. In the following discussion, I hope to highlight some of the more notable differences from the book, as well as draw some attention to new concepts not discussed in the book.

\subsection{Direct Sums}
Direct sums are analogous to bases. Whereas a basis allows for the unique representation of vectors in terms of the spans of the basis vectors, a direct sum of vector space allows for the unique representation of a vector by choosing one vector from each vector space.

Let $U, V$ be 2 vector subspaces of some vector space $W$.
The following statements are equivalent.
\begin{enumerate}
	\item $u_1 + v_1 = u_2 + v_2 \implies u_1=u_2 \land v_1=v_2$
	\item $u + v = 0\implies u=v=0$
	\item $U\cap V=\{0\}$
\end{enumerate}
Hence any of the 3 statements can be used to define $U\oplus V$. In the case of MA2101, no. 3 is used.

\subsubsection{Multiple direct sums}
Prof Brett defines
\begin{align*}
	\oplus_{i=1}^n U_i = (\dots((U_1 \oplus U_2)\oplus U_3) \oplus \dots)\oplus U_n
\end{align*}
In particular, $U_1 \oplus U_2\oplus U_3=(U_1 \oplus U_2)\oplus U_3$

It turns out that it doesn't matter which order to conduct the direct sum.

The following definition of a multiple direct sum (i.e. direct sum of more than 2 vector spaces) is perhaps more enlightening.\\
Suppose $\sum_{i=1}^n U_i = \oplus_{i=1}^n U_i$.\\
Then $\sum_{i=1}^n u_i=0\implies \forall 1\leq i\leq n, u_i=0$

With this definition in mind, suppose $(U_1 \oplus U_2)\oplus U_3$, and let $u_1 + u_2 + u_3 = 0$. This implies $(u_1 + u_2) + u_3 = 0$, where $u_1+u_2\in U_1+U_2, u_3\in U_3$. Definition of direct sum for 2 vector spaces says that $u_1+u_2=u_3=0$.\\
Now, this reduces to $u_1+u_2=0$. Since $U_1 \oplus U_2$, we then have $u_1=u_2=0$.

Hence, it makes sense to define multiple direct sums in terms of "single" direct sums (i.e. direct sums of 2 vector spaces).

\subsection{Vector space isomorphisms}
\textbf{Definition} A vector space homomorphism is a linear transformation.

A vector space isomorphism is a bijective linear map.

It is here that the notion of finite-dimensionality is introduced. Note that it is perfectly fine to define finite-dimensional without dimension. We simply need to capture the idea of finiteness.\\
In Axler's book, a finite-dimensional vector space $V$ is one which has a finite spanning set. i.e. A finite subset of $V$ exists that spans $V$.

Prof Brett defines finite-dimension via isomorphisms. A finite-dimensional vector space $V$ is one that is isomorphic to $\mathbb{F}^n$ for some $n\in \mathbb{N}$.\\
Contrary to the book's definition, it is immediately (intuitively) obvious that such an $n$ is unique, if it exists. Suppose not, such that $V$ is isomorphic to both $F^m$ and $F^n$, $m\neq n$. Then since the composition of bijective linear maps gives another bijective linear map, $\mathbb{F}^m\cong \mathbb{F}^n$, which immediately seems absurd. (The rigorous proof for this however, probably still requires the Replacement Theorem. So on a rigorous basis, this probably isn't much easier than Axler's approach.)


\subsection{Bases as isomorphisms from $\mathbb{F}^n$ to $V$}
In this course, the convention is to use basis vectors starting from the end of the alphabet. Occasionally, I will also use $u,v$.

Given a basis $z=(z_1,\dots,z_n)$ of $V$, we can also define the mapping 
\begin{align*}
z:\mathbb{F}^n\rightarrow V\\
z(a)=\sum_{i=1}^na_iz_i
\end{align*}
where $a=(a_1,\dots, a_n)\in \mathbb{F}^n$. It is easy to verify that $z$ is a bijective linear map, i.e., $z$ is a vector space isomorphism.

\textbf{Proposition} The range/image of a linear transformation $T:U\rightarrow V$ is the span of $T$ applied to the basis vectors of $U$.

\textbf{Proposition} A vector space isomorphism $\phi: \mathbb{F}^n\rightarrow V$ admits a basis $(\phi(e_1),\dots, \phi(e_n))$.

\textbf{Proof} Since linear map $\phi$ is injective, the vectors $(\phi(e_1),\dots, \phi(e_n))$ span $V$. Since $\phi$ is injective, $\sum_{i=0}^n a_i\phi(e_i)=0\implies \phi(\sum_{i=0}^n a_ie_i)=0 \implies \sum_{i=0}^n a_ie_i=0 \implies \forall i, a_i = 0$, which proves linear independence. Hence we have a basis.

Hence, "there is a one-one correspondence between isomorphisms of the form $\mathbb{F}^n\rightarrow V$ and bases".

Here is another way to think about a basis $z=(z_1,\dots,z_n)$ of $V$ . In Axler's book, the matrix representation of a vector $v\in V$ with respect to a basis $z$ is denoted as $M(v, z)$. So if we define the linear map $M(\cdot, z): V\rightarrow \mathbb{F}^{n,1}\cong \mathbb{F}^n$, we then notice that this is the inverse function of $z:\mathbb{F}^n\rightarrow V$.


\subsection{Dimension}
In Axler's book, dimension of a vector space is defined to be the number of vectors in a basis of $V$. This is well-defined since by the Replacement Theorem, the number of vectors in any basis of a vector space is the same.

In MA2101, as mentioned previously, defining a vector space $V$ to be  finite-dimensional if it is isomorphic to $\mathbb{F}^n$ for some $n\in \mathbb{N}$ suggests a natural definition for dimension -- the value of $n$. (Of course, as mentioned previously, a rigorous treatment of this would still require proving that this $n$ is unique if it exists.) Now, we take this for granted, that $n$ is unique.

\textbf{Proposition} The dimension of a finite-dimensional vector space $V$ is equal to the number of vectors in any basis.

\textbf{Proof} Consider a basis $z=(z_1,\dots,z_n)$ of $V$. $z$ then corresponds to some vector space isomorphism $\phi_z:\mathbb{F}^n\rightarrow V$ as defined above. (The vectors of basis $z$ are then given by $\phi_z(e_1),\dots,\phi_z(e_n)$.) By definition of dimension, this precisely says that the dimension of $V$ is $n$.

\section{Linear maps}

\subsection{Change of basis}
As mentioned in the previous section, for any basis $z:\mathbb{F}^n\rightarrow V$, we have $z^{-1}=M(\cdot,z):V\rightarrow \mathbb{F}^n$.

Let $y,z$ be 2 bases of vector space $V$.\\
Let $P=z^{-1}\circ y:\mathbb{F}^n\rightarrow \mathbb{F}^n$.\\
We then have $y=z\circ P$. Here, $P$ is a change of basis from $z$ to $y$ (\textbf{\textit{note the order!}}). In the notation of Axler's book, $P=M(I, y, z)$. ($I$ is the identity linear map.) For any $v\in V$, $M(I,y,z)M(v,y)=M(v,z)$.

Another way to remember the bolded part above is to think of the change of basis of a linear operator $T\in L(V)$.
\begin{align*}
	M(T, y, y) = M(I, z, y)M(T, z, z)M(I, y, z) = P^{-1}M(T, z, z)P
\end{align*}
Hence $P$ changes the basis on which $T$ acts from $z$ to $y$.

Given a certain basis $z$ of $V$, every other basis can be expressed in the form of $z\circ P$ for some isomorphism $P:\mathbb{F}^n\rightarrow \mathbb{F}^n$.

\subsection{Dual spaces}
Given a vector space $V$ over $\mathbb{F}$. The dual space of $V$, the set of all linear functionals on $V$, is denoted $V'=\hat{V}=L(V, \mathbb{F})$.

\subsection{Kernel and range}
In MA2101, the null space/kernel of a linear map $T:V\rightarrow W$ is denoted $ker(T)$. $dim\, ker(T) = null(T)$

The range of $T$ is denoted $rang(T)$. $dim\, rang(T)=rank(T)$

In Axler's book, rank is actually defined in terms of the column/row rank of a matrix representation of $T$. Then, rank is proven to be equal to $dim\, rang(T)$ by establishing an isomorphism between the column space of $M(T,B1,B2)$ and the range of $T$.

\subsubsection{Fundamental Theorem of Linear Maps}
Given linear map $T:V\rightarrow W$,
\begin{align*}
	dim V = null(T) + rank(T)
\end{align*}

There are 2 ways to prove the dimension theorem.
The first way (Axler) is to construct a basis $B_{null(T)}=\{y_1,\dots, y_k\}$ of $ker(T)$, then extend it to a basis $B_V=\{y_1,\dots,y_k, z_1,\dots, z_m\}$ of $V$. Show that $(Tz_1,\dots, Tz_m)$ is a basis for $range(T)$. 
Hence,
\begin{align*}
	dim(V) = k + m = null(T) + rank(T)
\end{align*}

The second way (MA2101) is to find a subspace $U$ of $V$ such that $V=null(T)\oplus U$. It is possible to find such a subspace via basis extension. Show that $U\cong range(T)$ by considering the restriction of $T$ to $U$ (also restrict codomain $V$ to $range(T)$. 
\begin{align*}
	T|_U:U\rightarrow range(T)
\end{align*}
Since isomorphism by definition (in MA2101) implies equal dimension, we are done.

\subsection{Dual vectors}
\textbf{Proposition} Given a basis $z=(z_1,\dots,z_n)$ of a vector space $V$, and any choice of vectors $w_1,\dots, w_n\in W$, a linear map $T:V\rightarrow W$ is uniquely identified by $T(z_i)=w_i, i\in \{1,\dots,n\}$.

The proof is trivial.

\textbf{Corollary} Given a basis $z=(z_1,\dots,z_n)$ of a vector space $V$, and linear maps $T_1,T_2:V\rightarrow W$. To check if $T_1=T_2$, it suffices to check that $\forall i\in \{1,\dots,n\}, T_1(z_i)=T_2(z_i)$

Hence we can define dual basis vectors as follows.\\
\textbf{Definition} $\forall i,j\in \{1,\dots, n\}, \zeta_i(z_j)=\delta_{ij}$.

Prof Brett uses Einstein's notation, however, I decide to stick with the classical subscript notation. However, I will still follow certain conventions for this course, e.g. scalars start from the start of the alphabet; basis vectors start from the end of the alphabet.

To show that $\zeta$ is indeed a basis for $\hat{V}$, we need to check for linear independence and span. Linear independence is easy. As for span, for any $\alpha \in \hat{V}$, define $\beta=\sum_i(\alpha z_i)\zeta_i$. Using the corollary above, it suffices to check for $\forall j\in \{1,\dots,n\}, \alpha z_j=\beta z_j$. Hence $\alpha=\beta$ and $\zeta$ spans $\hat{V}$.

\textbf{Proposition} $\alpha\sum_ip_i\zeta_i\iff p_i=\alpha z_i$

The backward direction is proven above. The forward direction is proven by applying $\alpha$ to $z_j$ for each $j=1,\dots,n$.

Now, temporarily forget the original set-based definition of the dual basis. We want to see if we can define the dual basis using an isomorphism, i.e. a map-based definition.

So now we don't have a general definition of dual basis. However, we still wish to define a special case for the dual space of $\mathbb{F}^n, \hat{\mathbb{F}^n}$. The dual basis vectors of $\hat{\mathbb{F}^n}$ are the row vectors $(\epsilon_1,\dots \epsilon_n)$, where $\epsilon_i=(e_i)^T$.

Based on this special case, we now define the following:\\
\textbf{Definition} Given a basis $z$ of $V$, the dual basis $\zeta$ of $\hat{V}$ is the isomorphism $\zeta: \hat{\mathbb{F}^n}\rightarrow \hat{V}$ such that 
\begin{align*}
	\forall a\in \mathbb{F}^n,\, \forall p\in \hat{\mathbb{F}^n}, (\zeta p)(za)=pa
\end{align*}
 
Now, I want to recover the original set-based definition of a dual basis. Let $p=\epsilon_i, a=e_j$, then $pa=\delta_{ij}$.\\
By definition $(\zeta \epsilon_i)(z_j)=\delta_{ij}$. Since $\zeta$ is an isomorphism, it corresponds to a set-basis with $(\zeta_1,\dots,\zeta_n)$, where $\zeta_i=\zeta \epsilon_i$. Hence, $\zeta_iz_j=\delta{ij}$.\\
And we have recovered the original set definition of a dual basis.

\subsubsection{Vector acting on dual vector}
I introduce the following notation:\\
Given $v\in V, \alpha \in \hat{V}$,
\begin{align*}
	\alpha(v)=\alpha v=v[\alpha]
\end{align*}
In MA2101, Prof Brett does not use square brackets, instead just writing $v(\alpha)=v\alpha$, however I find this to be potentially confusing.\\
The motivation for writing this will come later (e.g. when showing that the matrix representation of the transpose/dual map $\hat{T}$ of $T$ is really the transpose matrix of $T$).

\textbf{Definition} Given $V, W$, $T\in L(V, W)$, the \textbf{dual map}(Axler)/\textbf{transpose}(MA2101) of $T$ is denoted by $\hat{T}\in L(\hat{W},\hat{V})$. 
\begin{align*}
	\forall \alpha \in \hat{W}, \hat{T}(\alpha)=\alpha \circ T
\end{align*}
Hence, $\forall v\in V, v[\hat{T}\alpha]=(\alpha \circ T)v=\alpha Tv$. Note that the convention of linear maps is to forego the use of round brackets.\\
Notice the symmetry here.

\subsection{Tensor products}
\textbf{Definition} The tensor product $\otimes$ is a bilinear map defined as follows:
\begin{align*}
	\otimes: &V \times \hat{V}\rightarrow L(V)\\
	&(v, \alpha) \mapsto \alpha(\cdot)v
\end{align*}
That is, for each $w\in V$, $(v\otimes \alpha)w=\alpha(w)v$, multiplying $v$ by scalar $\alpha w$.

The transpose/dual map of $T=(v\otimes \alpha)$ has the following "nice" property. \\
For all $\beta \in \hat{V}, \hat{T}\beta=\beta \circ (v\otimes \alpha)\in \hat{V}$. Applying this to any arbitrary $w\in V$, we get 
\begin{align*}
	(\beta \circ (v\otimes \alpha))w&=\beta(\alpha(w)v)\\
	&=\alpha(w)\beta(v) && \text{linearity of $\beta$}\\
	&=\beta(v)\alpha(w) && \text{commutativity of scalar multiplication}\\
	&=(\beta(v)\alpha)(w) && \text{scalar multiplication of functions}
\end{align*}
Hence, $\hat{T}\beta=(\beta \circ (v\otimes \alpha))=\beta(v)\alpha$

\subsubsection{Basis of $L(V)$}
We are given that $z=(z_i)_{1\leq i\leq n}$ is a basis of $V$.\\
Let $\zeta=(\zeta_i)_{1\leq i\leq n}$ be the dual basis of $\hat{V}$.\\
Consider the following set 
\begin{align*}
	B=\{z_i\otimes \zeta_j : i,j\in \{1,\dots, n\}\}
\end{align*}
We claim that $B$ is a basis for $L(V)$.

\textbf{Claim}: $B$ spans $L(V)$.

Let $T\in L(V)$.\\
Define the scalars $T_{i,j}, i,j\in \{1,\dots,n\}$ as follows: For each $1\leq j\leq n$, $T(z_j)=\sum_{1\leq i\leq n}T_{i,j}z_i$.

Define 
\begin{align*}
	S=\sum_{1\leq i,j\leq n}T_{i,j}z_i \otimes \zeta_j
\end{align*}
Then for each $1\leq k\leq n$, 
\begin{align*}
	S(z_k)&=\sum_{1\leq i,j\leq n}T_{i,j}\zeta_j(z_k)z_i \\
	&=\sum_{1\leq i,j\leq n}T_{i,j}\delta_{jk}z_i\\
	&=\sum_{1\leq i\leq n}T_{i,k}z_i && \text{Substitute $j=k$}\\
	&=Tz_k && \text{definition of $T_{ij}$}
\end{align*}

Since linear maps are uniquely determined by their actions on basis vectors, $S=T$. This says that $B$ spans $L(V)$.

\textbf{Claim}: $B$ is linearly independent.

Suppose there exists $\lambda_{i,j}, 1\leq i,j\leq n$ such that \begin{align*}
	\sum_{1\leq i,j\leq n}\lambda_{i,j}z_i \otimes \zeta_j=0_{L(V)}
\end{align*}
Then, for each $1\leq k\leq n$, 
\begin{align*}
	0_V=0_{L(V)}(z_k)&=\sum_{1\leq i,j\leq n}\lambda_{i,j}\zeta_j(z_k)z_i \\
	&=\sum_{1\leq i,j\leq n}\lambda_{i,j}\delta_{jk}z_i\\
	&=\sum_{1\leq i\leq n}\lambda_{i,k}z_i \\
\end{align*}
Since $z$ is a basis, it is a linearly independent set, hence $\forall 1\leq i\leq n,\, \lambda_{i,k}=0$. $k$ runs from $1$ to $n$, so we have $\lambda_{i,k}=0$ for all $1\leq i,k\leq n$. This proves linear independence of $B$. 

\textbf{Definition} $T_{i,j}$ are called the components of $T\in L(V)$ with respect to basis $z$.

Before, the definition of $T_{i,j}$ is implicitly given by $T(z_j)=\sum_{1\leq i\leq n}T_{i,j}z_i$. Here is a more explicit way to express an individual entry of $T_{i,j}$.
\begin{align*}
	\forall 1\leq i,j\leq n,\, T_{i,j}=\zeta_{i}(Tz_j)
\end{align*}

\subsection{Matrices}
In MA2101, the $(i,j)$-th entry of a $n$ times $m$ matrix $M\in \mathbb{F}^{n,m}$ is denoted as $M^i_j$.
I will use the "classical" $M_{i,j}$ instead.\\
If the entries of a matrix $M_{i,j}$ are known and equal $a_{i,j}$, then $M$ can be denoted as $(a_{i,j})$.

All the matrix notation here onwards will largely follow Axler's book.

\textbf{Proposition} Given $M\in \mathbb{F}^{n,m}, N\in \mathbb{F}^{m,p}$,
\begin{align*}
	MN = \sum_{1\leq i\leq m}M_{\cdot,i}N_{i,\cdot}
\end{align*}

This is covered in MA1101R and is one of the multiple ways with which to view matrix multiplication.

Given a linear operator $T\in L(V)$, we know from the previous section that $T=\sum_{1\leq i,j\leq n}T_{i,j}(z_i\otimes \zeta_j)$.
Its matrix representation with respect to the basis $z$ is $(T_{i,j})=(\zeta_i(Tz_j))$.

\textbf{Theorem} The matrix representation of $\hat{T}$ with respect to basis $\zeta$ is the transpose of the matrix representation of $T$ w.r.t basis $z$.

\textbf{Remark} Prof Brett actually refers to the matrix representation of $\hat{T}$ relative to basis $z$, since $\zeta$ is derived from $z$. However, personally, I'd like to emphasise that $\hat{T}$ maps from $\hat{V}$ to $\hat{V}$, so when referring to the basis I also would refer to the dual basis of $\hat{V}$.

\textbf{Proof}
We have
(1)
\begin{align*}
	T_{i,j}=\zeta_i(Tz_j)=(\zeta_i\circ T)z_j=(\hat{T}\zeta_i)z_j=z_j[\hat{T}\zeta_i]
\end{align*}
Notice that as $\hat{T}$ is just another linear operator, we have already defined $\hat{T}_{i,j}$ implicitly in the previous section as 
\begin{align*}
	\hat{T}\zeta_j=\sum_{1\leq i\leq n}\hat{T}_{i,j}\zeta_i
\end{align*}
Hence, 
(2)
\begin{align*}
	z_k[\hat{T}\zeta_i]&=\sum_{1\leq i\leq n}\hat{T}_{i,j}\zeta_i(z_k)\\
	&=\hat{T}_{k,j}
\end{align*}
Hence, combining (1) and (2), $\forall 1\leq i,j\leq n$,
\begin{align*}
	T_{i,j}=z_j[\hat{T}\zeta_i]=\hat{T}_{j,i}
\end{align*}

Hence, in matrix form, $(\hat{T}_{i,j})=(T_{i,j})^t$.


\subsection{Linear maps $L(V,W), V\neq W$}
As a convention, let $dim(V)=n, dim(W)=m$
Define the tensor product:
\begin{align*}
	\otimes: &W \times V' \rightarrow L(V,W)\\
	&(w, \alpha) \mapsto \alpha(\cdot)w
\end{align*}
i.e. $\forall v\in V, (w \otimes \alpha)v = \alpha(v)w$

This definition makes a lot more sense when we try to define the basis of $L(V,W)$ in terms of the tensor product.

Let $z=(z_1,\dots, z_n)$ be a basis for $V$.
Let $x=(x_1,\dots, x_m)$ be a basis for $W$.

\textbf{Proposition} The set 
\begin{align*}
	\{x_i \otimes \zeta_j : 1\leq i\leq m, 1\leq j\leq n\}
\end{align*}
is a basis for $L(V,W)$.


\end{document}




