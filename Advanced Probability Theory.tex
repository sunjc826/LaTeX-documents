\documentclass[a4paper]{article}
\usepackage[english]{babel}
\usepackage[utf8]{inputenc}
\usepackage{amsmath,amssymb}
\usepackage{parskip}
\usepackage{graphicx}
\graphicspath{ {./images/} }
\usepackage{gensymb}
\usepackage{listings}
\usepackage{caption}
\usepackage{pgfplots}

\newcommand\tab[1][1cm]{\hspace*{#1}}
\newcommand{\powerset}[1]{\mathcal{P}(#1)}
\newcommand{\reals}[0]{\mathbb{R}} %real numbers%
\newcommand{\alg}[0]{\mathcal{F}} %sigma algebra%
\newcommand{\borel}[0]{\mathcal{B}} %Borel algebra%
\newcommand{\measure}[0]{\mathcal{M}}
\newcommand{\semialg}[0]{\mathcal{J}} %semi algebra%
\newcommand{\triple}[0]{(\Omega, \alg, P)} %probability triple%
\newcommand{\real}[0]{\mathbb{R}} %real numbers%
\newcommand{\rational}[0]{\mathbb{Q}} %rational numbers%
\newcommand{\integer}[0]{\mathbb{Z}} %integers%
\newcommand{\nat}[0]{\mathbb{N}} %natural numbers%
\newcommand{\limitinf}[1][n]{\lim_{#1\rightarrow \infty}} %limit to infinity%
\newcommand{\forallReal}[1][x]{\forall #1\in\real} 
\newcommand{\forallInteger}[1][n]{\forall #1\in\integer}
\newcommand{\forallNat}[1][n]{\forall #1\in\nat}
\newcommand{\existsReal}[1][x]{\exists #1\in\real}
\newcommand{\existsInteger}[1][n]{\exists #1\in\integer}
\newcommand{\existsNat}[1][n]{\exists #1\in\nat}
\newcommand{\seq}[2]{#1_1,#1_2,\dots,#1_{#2}} %finite sequence literal%
\newcommand{\infseq}[1]{#1_1,#1_2,\dots} %infinite sequence%
\newcommand{\subseq}[3]{#1_{#2_1},#1_{#2_2},\dots,#1_{#2_{#3}}} %finite subsequence literal%
\newcommand{\seqconn}[3]{#1_1 #3 #1_2 #3 \dots #3 #1_{#2}} %sequence delimited by argument 3%
\newcommand{\subseqconn}[4]{#1_{#2_1} #4 #1_{#2_2} #4 \dots #4 #1_{#2_{#3}}} %subsequence literal%

% Margins
% top=2.54cm, left=2.54cm, right=2.54cm, bottom=2.54cm
\usepackage[a4paper, top=2.54cm, left=2.54cm, right=2.54cm, bottom=2.54cm]{geometry}
\usepackage{hyperref}
\hypersetup{
    colorlinks=true,
    linkcolor=blue,
    filecolor=magenta,      
    urlcolor=blue,
}

\title{Advanced Probability Theory}
\author{Jia Cheng}
\date{December 2021}

\begin{document}
\maketitle

Reference Text: A First Look at Rigorous Probability Theory

Related modules: ST5214

\paragraph{Notation} Throughout this document, $\mathbb{N}$ can vary in usage, sometimes $\mathbb{N}=\mathbb{Z}^+$, other times $\mathbb{N}=\mathbb{Z}^+_0$, depending on which is more convenient.

\paragraph{1.1}

To see that $\exists z\in \mathbb{R}, P(Z=z) > 0$, we observe that
\[P(Z=0)\geq P(Z=0\land X=0) = P(Z=0|X=0)\cdot P(X=0)\geq \frac{1}{2}\cdot P(X=0) > 0\]
since $P(X=0) > 0$.


\paragraph{1.2}

\subparagraph{Uncountable summation} Given an uncountable non-negative set of numbers $\{r_a : a\in I\}$ indexed by $I$,

\[\sum_{a\in I}r_a := \sup \{\sum_{a\in J}r_a : J\subseteq I\land J \text{ finite}\}\]

\subparagraph{R-shift}
(Equivalent definition) R-shift of $A\subseteq [0, 1]$. $A\oplus r = \{(a + r)\mod 1 : a\in A\}$


\paragraph{2.1}

Notice that countability is used by 2 constructs. One, probability measure is countably additive (and not uncountably so). Two, the $\sigma$-algebra is closed under countable union and intersection (and not uncountably so).

Recall that the reason for disallowing uncountable operations in general is due to the fact that
\[\bigcup_{x\in A}\{x\} = A\]
for any set $A$, in particular $[0,1]$ when discussing the uniform distribution on the unit interval.

\paragraph{Theorem 2.2.1} We provide a proof for this theorem.

First, we show that $\mathcal{F}$ is a $\sigma$-algebra. By definition, $\mathcal{F} = \powerset{\Omega}$. Hence, the unary complement operation is a mapping $\mathcal{P}(\Omega)\rightarrow \mathcal{P}(\Omega)$ whose domain and codomain are just $\mathcal{F}$ as desired. Similarly, for the countable set operations of union and intersection, they are mappings with codomain as $\powerset{\Omega}$ and are also closed since $\mathcal{F} = \powerset{\Omega}$.\\
We also note that both $\emptyset, \Omega$ reside in $\mathcal{F}$

Next, we show that $P$ is a probability measure. By definition of $P$, $P$ is additive since $A\cap B=\emptyset \implies P(A\sqcup B) = \sum_{\omega \in A\sqcup B}p(\omega) = \sum_{\omega \in A}p(\omega) + \sum_{\omega \in B}p(\omega) = P(A) + P(B)$. 

I am not quite sure about showing countable additivity however, perhaps using some form of diagonal summation argument it is possible to prove this.

$P(\Omega) = \sum_{\omega \in \Omega}p(\omega) = 1$. Furthermore, $p$ is non-negative, hence $P$ is indeed bounded between $0$ and $1$.

\paragraph{Ex 2.2.3} First, $\emptyset, \Omega = [0,1] \in \semialg$ by definition as they are intervals. Next, to show closure under finite intersection, it suffices to show closure under binary intersection. Consider cases: We only consider one endpoint, since we can "patch" together two endpoints.
\begin{itemize}
	\item $[a$ and $[b$ intersect to give $[\max\{a,b\}$
	\item $(a$ and $(b$ intersect to give $(\max\{a,b\}$
	\item $[a$ and $(b$ intersect to give $[a$ if $a > b$ and $(b$ otherwise
\end{itemize}
We can do a similar case analysis for right endpoints.
Given a stringified left endpoint $l\in \{ "[a", "(a" \}$ and a stringified right endpoint $r\in \{ "b]", "b)" \}$ we can form an interval via concatenation $l,r$.
Hence, $\semialg$ is closed under finite intersection.

Consider the complement $J = [0, a)\cup (b, 1]$ of an interval $[a,b]\in \semialg$, where depending on whether the left/right endpoint is closed or open we adjust $J$ accordingly. Regardless, we see that $J$ is a disjoint union of at most 2 intervals in $\semialg$.

Hence, $\semialg$ is a semialgebra of subsets of $\Omega$.

\paragraph{Ex 2.2.5} 
\subparagraph{a} $\mathcal{B}_0\subseteq \powerset{\Omega}$. Since $\mathcal{B}_0$ consists of all finite unions of elements of $\semialg$, in particular, $\semialg \subseteq \mathcal{B}_0$, so $\empty, \Omega = [0,1]\in \semialg \subseteq \mathcal{B}_0$.

Next, the finite union and intersection of elements of $\mathcal{B}_0$ will give finite unions of elements of $semialg$, so that $\mathcal{B}_0$ is closed under finite union and intersection. (For intersection, we can argue using distributive law plus observe that the intersection of intervals gives another interval)

Let $B\in \mathcal{B}_0$, so that $B$ is a finite union of the form $\bigcup_{1\leq i\leq n}I_i$ for some intervals $I_i$ in $[0,1]$. Then, $B^c = \bigcap_{1\leq i\leq n}I_i^c$ by DeMorgan's Law, and we have already proven in Ex 2.2.3 that $I_i^c$ is a disjoint union of intervals, i.e. $I_i^c\in \mathcal{B}_0$. Furthermore, we have proven that $\mathcal{B}_0$ is closed under finite intersection, so $B^c\in \mathcal{B}_0$.

Hence, $\mathcal{B}_0$ is an algebra.

\subparagraph{b} The difference between an algebra and a $\sigma$-algebra is that $\sigma$-algebras are closed under countable union and intersection but algebras are not necessarily so.

We consider Cantor's set $C$, which is a countable intersection of $C_i$, where each $C_i$ is formed by removing from each interval in $C_{i-1}$ the middle one-third.

Since each $C_i$ is a union of (disjoint) intervals, by definition, $C_i\in \mathcal{B}_0$. If $\mathcal{B}_0$ is to be a $\sigma$-algebra, then we must have $C\in \mathcal{B}_0$, i.e. $C$ can be formed from a finite union of intervals.

First of all, $C$ does not contain any interval of non-zero length, so our options are reduced to forming $C$ from a finite union of singletons, i.e. intervals of the form $[a,a]=\{a\}$. But we also know $C$ to be uncountable, but a finite union of singletons is finite. Hence we have a contradiction.

Since $C\notin \mathcal{B}_0$, $\mathcal{B}_0$ is not closed under countable intersection, so it is not a $\sigma$-algebra.


We comment that $\mathcal{B}_1$ is similarly not a $\sigma$-algebra using the same counterexample. The countable union of singletons must be at most countable, so they cannot union to form an uncountable set like $C$.

\paragraph{Ex 2.3.16} Suppose $A\in \mathcal{M}\land P^*(A) = 0$ and $B\subseteq A$. To show that $B\in \mathcal{M}$, we show equivalently that:

For each $E\subseteq \Omega$, $P^*(E)\geq P^*(A\cap E) + P^*(A^c\cap E)$ (i.e. superadditivity. But by monotonicity, $P^*(A\cap E) = 0$, so that $P^*(E)\geq P^*(A^c\cap E)$ is automatically true by monotonicity. Hence, we have shown the completeness of the extension $(\Omega, \mathcal{M}, P^*)$.


See \url{https://mathoverflow.net/questions/11554/whats-the-use-of-a-complete-measure}\\
\textbf{Remark} A complete measure is good in the sense that it gives us more measurable sets, the more things we can measure, the better. For instance, one consequence of a complete measure is that when $A$ is measurable, and $B$ differs from $A$ by a subset of a set of zero measure, then $B$ is measurable as well.

\paragraph{Ex 2.4.3}
\subparagraph{a} We make the following manipulations.
\[
	I\subseteq \bigcup_{1\leq j\leq n}I_j = \sqcup_{1\leq j\leq n}I'_j
\]
where $I'_j = I_j - \bigcup_{1\leq l < j}I_l$. Hence,
\begin{align*}
	I = I\cap \sqcup_{1\leq j\leq n} I'_j = \sqcup{1\leq j\leq n} (I\cap I'_j)
\end{align*}
such that
\begin{align*}
	P(I) &= P(\sqcup{1\leq j\leq n} (I\cap I'_j))\\
	&=\sum_{1\leq j\leq n}|I\cap I'_j|\\
	&\leq \sum_{1\leq j\leq n} |I'_j|\\
	&\leq \sum_{1\leq j\leq n} |I_j|\\
	&=\sum_{1\leq j\leq n} P(I_j)
\end{align*}
where the first equality is proven in proposition 2.4.2 (disjoint union of intervals to form a single larger interval), and the latter inequalities are given by monotonicity of the length function $|\cdot|$.

\subparagraph{b}
In $[0,1]\subseteq \mathbb{R}$, $I$ is a closed and bounded interval, hence compact (Heine-Borel Theorem). Given countable cover $I_i, i\in \mathbb{N}$, there exists a finite subcover of $I$, i.e. $I\subseteq \bigcup_{1\leq j\leq n}I_{i_j}$. By part (a), we know that $P(I)\leq \sum_{1\leq j\leq n}P(I_{i_j})\leq \sum_{j}P(I_j)$ since $P$ is non-negative.

\subparagraph{c} We want to generalize our result in (b), i.e. $I$ can be any interval, not just closed, and $I_j$ can be any interval, not just open.

We extend each $I_j, j\geq 1$ to form $I'_j := (a_j - \epsilon 2^{-j-1}, b_j + \epsilon 2^{-j-1})$, such that $|I'_j| = |I_j| + \epsilon2^{-j}$.

We compress $I$ to form $I':=[a+\epsilon, b-\epsilon]$. This assumes $I\neq \emptyset$, since if $I$ is empty, we trivially have $P(I)=0\leq \sum_j P(I_j)$.

We note that each $I'_j$ may very well exceed the boundaries of $[0,1]$, such that $P$ may not be defined for $I'_j$, but it doesn't matter, since the length function $|I'_j|$ is still well-defined.

Hence, \begin{align*}
	P(I) = |I| = |I'| + 2\epsilon
\end{align*}
and 
\begin{align*}
	|I'|\leq \sum_{j} |I'_j| = \sum_{j} |I_j| + \epsilon = \sum_{j} P(I_j) + \epsilon
\end{align*}

Combining gives
\begin{align*}
	P(I)\leq \sum_{j} P(I_j) + 3\epsilon
\end{align*}
which proves countable monotonicity in general for all sets $I, I_1, I_2,\dots \in \mathcal{J}$.


An unsuccessful attempt: I tried to use closure $I' := \overline{I} = [a,b]$, but it doesn't seem possible to show that the endpoints lie in some open interval.

\paragraph{Ex 2.4.5} Claim: $\sigma(\{(-\infty, b] : b\in \mathbb{R}\}) = \sigma(\{I\subseteq \mathbb{R}:I \text{ interval}\})$

We first note that for collections $\mathcal{A_1}, \mathcal{A_2}$, $\mathcal{A_1}\subseteq \mathcal{A_2}\implies \sigma(\mathcal{A_1})\subseteq \sigma(\mathcal{A_2})$, since a Borel algebra that contains all elements of $\mathcal{A_2}$ also contains all elements of $\mathcal{A_1}$.

Let $\mathcal{A_1} = \{(-\infty, b] : b\in \mathbb{R}\}$ and $\mathcal{A_2}$ be the set of all intervals in $\mathbb{R}$. Let $\sigma_1 = \sigma(\mathcal{A_1}), \sigma_2 = \sigma(\mathcal{A_2})$. Clearly, $\sigma_1\subseteq\sigma_2$.

Next, we make the following 4 steps
\begin{enumerate}
	\item First, for each $b$, $(-\infty, b]\in \sigma_1$ by definition
	\item By closure under complements, for each $a$, $(-\infty, a]^c = (a, \infty)\in\sigma_1$
	\item By closure under countable union, $\bigcup_{n\in \mathbb{N}}(-\infty,b-\frac{1}{n}] = (-\infty, b)\in\sigma_1$
	\item By closure under countable intersection $\bigcup_{n\in \mathbb{N}}(a-\frac{1}{n}, \infty) = [a, \infty)\in\sigma_1$
\end{enumerate}
Now we observe that $\forall a,b\in\mathbb{R},a\leq b,$
\begin{itemize}
	\item $[a,b] = [a, \infty)\cap (-\infty, b]$
	\item $(a,b) = (a, \infty)\cap (-\infty, b)$
	\item $[a,b) = [a, \infty)\cap (-\infty, b)$
	\item $(a,b] = (a, \infty)\cap (-\infty, b]$
\end{itemize}
hence by closure, $\sigma_1$ contains all intervals, i.e. $A_2\subseteq \sigma_1$. By minimality of Borel algebra $\sigma_2$, we have $\sigma_2\subseteq\sigma_1$. Hence $\sigma_1=\sigma_2$.

(We note that a Borel algebra of a set $A$ is a subset of any $\sigma$-algebra containing the same set $A$)

A Borel algebra $\borel = \sigma(A)$ reminds me of the subgroup $\langle S\rangle$ generated by a subset $S$ of a group. $\langle S\rangle$ is equal to the set of all words formed by the (finite) product of elements of $S$, whereas $\borel$ is the set of all countable unions, countable intersections, complements of elements of $A$.

\paragraph{Ex 2.4.7}
\subparagraph{a} $K$ is the countable intersection of intervals remaining at each step, whereas $K^c$ is the countable union of the intervals removed at each step of the construction. Hence, by closure of $\sigma$-algebras under countable union and intersection, $K, K^c\in \borel$.
\subparagraph{b} Since $\borel\subseteq\measure$, $K,K^c\in\measure$
\subparagraph{c} Algebras are closed under countable union, so $K^c\in \mathcal{B}_1$
\subparagraph{d} $K$ is an uncountable set. Furthermore, $K$ contains no interval of non-zero length. It is not possible to obtain $K$ with the countable union of singletons. Hence $K\notin \borel_1$.
\subparagraph{e} Since $\borel_1$ is not closed under complement, $\borel_1$ is not a $\sigma$-algebra. (We can also say that $\borel_1$ is not closed under countable intersection.)

\paragraph{2.5.4} To explicitly show the claim that $D_n = B_n\cap \bigcap_{1\leq i\leq n-1}B_i^c$ is a disjoint union of elements of $\semialg$, we let each $B_n^c = \sqcup_{1\leq j\leq l_n}F_{n,j}$ where each $F_{n,j}\in \semialg$.

Then
\begin{align*}
	B_n\cap \bigcap_{1\leq i\leq n-1}B_i^c = B_n\cap \bigcap_{1\leq i\leq n-1}\sqcup_{1\leq j\leq l_n}F_{n_j}
\end{align*}
By distributive law, this resolves to a union of terms of the form
$B_n\cap F_{1, \lambda_1}cap F_{2, \lambda_2}\cap \dots \cap F_{n-1, \lambda_{n-1}}$ where for each $1\leq i\leq n-1$, $1\leq \lambda_i \leq l_i$. Since each term is a finite  intersection, it resides in $\semialg$.

Any 2 distinct terms of the intersection, $B_n\cap F_{1, \lambda_1}\cap F_{2, \lambda_2}\cap \dots \cap F_{n-1, \lambda_{n-1}}$, and $B_n\cap F_{1, \mu_1}\cap F_{2, \mu_2}\cap \dots \cap F_{n-1, \mu_{n-1}}$ must have some $i$ for which $\lambda_i\neq\mu_i$. This implies that $F_{i, \lambda_i}\cap F_{i, \mu_i} = \emptyset$ so the 2 terms must also be disjoint.

\paragraph{Ex 2.5.6} Suppose $P$ satisfies finite additivity, and that given a monontonically decreasing set sequence $(A_i)$, with each $A_i$ being a finite disjoint union of elements of $\semialg$ and $\cap_n A_n = \emptyset$, that $\lim_{n\rightarrow \infty}P(A_n) = 0$. (Note that $P$ is extended to sets $A_n$ by finite additivity, mentioned in the errata document)

Suppose we have a countably many pairwise disjoint sets $D_n, n\in \mathbb{N}$. Then, we note that we can write
\begin{align*}
	\sqcup_i D_i = D_1 \sqcup D_2 \sqcup \dots \sqcup D_n \sqcup \left(\sqcup_{i>n}D_i\right) 
\end{align*}

We now show some properties of the tail union $\sqcup_{i>n}D_i$.
First, letting $A_n = \sqcup_{i>n}D_i = \sqcup_i D_i \cap D_1^c \cap D_2^c \cap \dots \cap D_n^c$. Similar to \textbf{2.5.4}, we can show that $A_n$ is a disjoint union of elements of $\semialg$.
Furthermore, $A_n\supseteq A_{n+1}$ and $\cap_n A_n = \emptyset$ are easy to verify. Hence, by assumption, $\lim_{n\rightarrow \infty}P(A_n) = 0$.

Hence, \begin{align*}
	P(\sqcup_i D_i) = \lim_{n\rightarrow \infty}(P(D_1) + \dots + P(D_n) + P(A_n)) = \lim_{n\rightarrow \infty}\sum_{1\leq i\leq n}P(D_i) = \sum_i P(D_i)
\end{align*}
as desired.

\paragraph{2.5.9} Let $S = \{(-\infty, x] : x\in\mathbb{R}\}$, $\semialg_1 = \{(-\infty, x] : x\in\mathbb{R}\}\cup\{(y,\infty):y\in\mathbb{R}\}\cup\{(y,x]:y,x\in\mathbb{R}\}\cup\{\emptyset,\mathbb{R}\}$, $\semialg_2$ be the set of all intervals on $\mathbb{R}$. Then $S\subseteq\semialg_1\subseteq\semialg_2$. (Note that $S$ is not a semialgebra but $\semialg_1$ is.)

Then $\sigma(S)\subseteq\sigma(\semialg_1)\subseteq\sigma(\semialg_2)$. But we have proven in \textbf{Ex 2.4.5} that $\sigma(S)=\sigma(\semialg_2)$, hence $\sigma(\semialg_1)=\sigma(\semialg_2)$.


\paragraph{Ex 2.6.1}
\subparagraph{a} First, we check that indeed, $\emptyset, \Omega\in\semialg$.\\
Next, we check that $\semialg$ is closed under finite intersection. 
\begin{itemize}
	\item If $\emptyset$ is part of an intersection, then we just get $\emptyset$.
	\item If $\Omega$ is part of an intersection, then we can just ignore it.
	\item Now consider $B_1\cap B_2$, where $B_1 = A_{a_1,a_2,\dots,a_m}, B_2 = A_{a'_1,a'_2,\dots,a'_n}$. We then consider cases. If WLOG, $a_1,\dots,a_m$ is a prefix of $a'_1,\dots,a'_n$, then $B_2\subseteq B_1$ and $B_1\cap B_2 = B_1\in\semialg$. If neither is a prefix of the other, then $B_1\cap B_2=\emptyset\in\semialg$.
\end{itemize}

Finally, consider any $B\in\semialg$. The case where $B=\emptyset\lor B=\Omega$ is trivial. So suppose $B = A_{a_1,\dots,a_n}$. Then $B^c$ is the finite disjoint union of $A_{a'_1,\dots,a'_n}$ where $\lor_{1\leq i\leq n} a'_i\neq a_i$. Hence, the complement of any sets of $\semialg$ can be formed by a finite disjoint union of sets in $\semialg$.

Hence $\semialg$ is a semialgebra.

\subparagraph{b} Suppose we have $N$ disjoint sets $\{D_1,\dots,D_n\}$.

Following the hint, letting $k\in \mathbb{N}$ be the largest number such that $a_1,a_2,\dots,a_k$ is a shared prefix amongst all sets $D_1$ to $D_n$.

Let $A=A_{a_1,\dots,a_k}$. Then, we clearly have $A\supseteq \bigcup_{1\leq i\leq n} D_i$. We claim that this is in fact an equality. Note that by assumption, the union of all $D_i$ is a set in the semialgebra $\semialg$, so the union must be of the form $A_{a_1,\dots,a_k,\dots,a_j}$ for some $j \geq k$. So we just want to show that $j=k$. Suppose not, such that $j>k$, then $a_1,\dots,a_j$ is a shared prefix, but this contradicts the largest-ness of $k$.

Hence, we have proven $\sqcup_{1\leq i\leq n}D_i = A$.

Next, we want to show finite additivity, i.e. $P(A) = \sum_{1\leq i\leq n}P(D_i)$. Let $K$ be the length of the longest sequence specified by $D_i, 1\leq i\leq n$. For e.g. if $D_1 = A_{1,0,1}, D_2 = A_{1,1}$, then the longest sequence is of length $K=3$ (and $k=1$).

We shall prove this by (finite) induction.

Define for each $m\in \{k,k+1,\dots,K\}$ the statement $Q(m)$:\\
Given any finite collection of disjoint sets $D_i$ whose longest specified sequence is of length $K$ and whose longest disjoint union is $A^m:=A_{a_1,\dots,a_m}$ for some length $m$ sequence, then $\sqcup_{1\leq i\leq n}P(D_i) = P(A^m)$.

The base case, $Q(K)$ is trivial since the finite collection can only consist of 1 set, $A^K$ itself. So we have $P(D_1) = P(A^K)$.

Next, suppose $Q(j)$ holds for some $k<j\leq K$. Then, we want to show $Q(j-1)$. We partition the set $D_i, 1\leq i\leq n$ into 2 subcollections, $B_1,\dots,B_{n_1}$ and $C_1,\dots,C_{n_2}$, such that each $B_i\subseteq A_{a_1,\dots,a_j,0}$ and each $C_i\subseteq A_{a_1,\dots,a_j,1}$. It is not too hard to argue that $\sqcup_{1\leq i\leq n_1} B_i = A_{a_1,\dots,a_j,0}$ and $\sqcup_{1\leq i\leq n_2} C_i = A_{a_1,\dots,a_j,1}$. Then, applying inductive assumption $Q(j)$, we have $\sum_{1\leq i\leq n_1}P(B_i) = P(A_{a_1,\dots,a_j,0})$ and $\sum_{1\leq i\leq n_2}P(C_i) = P(A_{a_1,\dots,a_j,1})$. Hence,
\begin{align*}
	P(A^j) = \frac{1}{2^j} = \frac{1}{2^{j+1}} + \frac{1}{2^{j+1}} = P(A_{a_1,\dots,a_j,0}) + P(A_{a_1,\dots,a_j,1}) = \sum_{1\leq i\leq n}P(D_i)
\end{align*}
This proves $Q(j)$.

By mathematical induction, we have $\forall m\in \{k,k+1,\dots,K\}, Q(m)$, in particular, $Q(k)$ is true. This completes the proof.


\paragraph{Ex 2.6.4} Clearly $\Omega = \Omega_1\times\Omega_2\in\semialg$ and $\emptyset=\emptyset\times\emptyset\in\semialg$.\\
Next, if $A_1\times B_1, A_2\times B_2\in\semialg$, then $(A_1\times B_1)\cap (A_2\times B_2) = (A_1\cap A_2)\times (B_1\cap B_2)\in \mathcal{F}_1\times\mathcal{F}_2 = \semialg$, so $\semialg$ is closed under finite intersection.\\
Finally, if $A\times B\in\semialg$, then $(A\times B)^c = (A^c\times \Omega_2)\cup (\Omega_1\times B^c) = (A^c\times B^c)\sqcup (A^c\times B)\sqcup (A\times B^c)$.\\
Note that $\mathcal{F}_1, \mathcal{F}_2$ are $\sigma$-algebra, so $A^c\in\mathcal{F}_1\land B^c\in\mathcal{F}_2$.
Hence $\semialg$ is a semialgebra of $\Omega_1\times\Omega_2$.

We also verify that $P(\emptyset) = P_1(\emptyset)P_2(\emptyset) = 0$ and $P(\Omega)=P_1(\Omega_1)P_2(\Omega_2) = 1$.

\paragraph{Ex 3.1.4}\mbox{}\\
Here, we use the property that $[a,b]\cap [c,d] = [\max\{a,c\},\min\{b,d\}]$.

We write $\omega$ as $w$. $Z(w) > a \iff 3w + 4 > a\iff w > \frac{a-4}{3}$. Hence, $P(Z > a) = P(\{w\in [0,1]:w>\frac{a-4}{3}\}) = P([0,1]\cap (\frac{a-4}{3}, \infty)) = P([0,1]\cap [\frac{a-4}{3},\infty)) = P([\max\{0,\frac{a-4}{3}\},1])$.
We note that $0 < \frac{a-4}{3} < 1\iff 4 < a < 7$, so
\begin{align*}
	P([\max\{0,\frac{a-4}{3}\},1]) = \begin{cases}
		1\, \text{if }a\leq 4\\
		1-\frac{a-4}{3}\, \text{if }4<a<7\\
		0\, \text{if }a\geq 7	
	\end{cases}
\end{align*}

Next, $\forall w\in \Omega = [0,1], X(w) < a\land Y(w) < b\iff w < a\land w < \frac{b}{2} \iff w < \min\{a,\frac{b}{2}\}$.
Hence, $P(X<a\land Y<b) = P((-\infty,\min\{a,\frac{b}{2}\})\cap [0,1]) = P((-\infty,\min\{a,\frac{b}{2}\}]\cap [0,1]) = P([0,\min\{1,a,\frac{b}{2}\}])$

\paragraph{3.1.5}\mbox{}\\ 
(ii) We elaborate that $\{X+Y < x\} = \bigcup_{a,b\in\real,a+b<x}\{X=a\land Y=b\}$, an uncountable union. However, choosing $r\in (a, x-b)\cap \rational$, we see that $\{X = a\land Y = b\}\subseteq \{X < r\land Y < x-r\}$, so the uncountable union $\bigcup_{a,b\in\real,a+b<x}\{X=a\land Y=b\}$ is indeed captured by the countable union $\bigcup_{r\in\rational}\{X < r\land Y < x-r\}$.

\paragraph{Ex 3.1.7} The goal is to show that
\begin{align*}
	\{Z\leq x\} = \bigcap_{m=1}^\infty\bigcup_{n=1}^\infty\bigcap_{k=n}^\infty \{Z_k\leq x + \frac{1}{m}\}
\end{align*}

For each $w\in \Omega$,
\begin{align*}
	&w\in \{Z\leq x\}\iff \\
	&Z(w) \leq x\iff \\
	&\limitinf Z_n(w) \leq x\iff \\
	&\forallNat[m], \existsNat[n], \forall k\geq n, Z_k(w) \leq  x + \frac{1}{m}\iff \\
	&\forallNat[m], \existsNat[n], \forall k\geq n, w\in \{Z_k \leq  x + \frac{1}{m}\}\iff \\
	&\land_{m\in \nat} \lor_{n\in \nat} \land_{k\geq n}, w\in \{Z_k \leq  x + \frac{1}{m}\}\iff \\
	&w\in \bigcap_{m=1}^\infty\bigcup_{n=1}^\infty\bigcap_{k=n}^\infty \{Z_k\leq x + \frac{1}{m}\}
\end{align*}
which proves LHS = RHS.

It is interesting to see the conversion from universal and existential quantifiers (commonly applied in analytical definition of limits) to boolean logic operators, and finally to set operators.

\paragraph{Proposition} (Restating the paragraph right below \textbf{ex 3.1.7}.) If $X$ is a random variable, and $f: \real\rightarrow\real$ is a Borel-measurable function, then the composition $f\circ X$ is also a random variable.

\paragraph{3.1.9} Suppose $\triple$ is complete. Let $X$ be a r.v. on $\Omega$. Let $Y: \Omega\rightarrow\real$ such that $P(X=Y)=1$. We claim that $Y$ is also a r.v. The proof is as follows.

We denote the event $E=\{X=Y\}$. Note that $P(E) = 1, P(E^c) = 0$.

For each $b\in\real$, $\{Y\leq b\} = (\{Y\leq b\}\cap E) \cup (\{Y\leq b\}\cap E^c)$, so that $P(Y\leq b) = P(Y\leq b\land X=Y) + P(Y\leq b\land X\neq Y) = P(Y\leq b\land X=Y) = P(X\leq b)$. (To be more precise, we should actually do the derivation backwards, starting with $P(X\leq b)$, since we don't know whether $\{Y\leq b\}\in \alg$ or not. But equality is symmetric.)\\
The last equality is by manipulation of sets. The second last equality is by completeness of $\triple$, and that $\{Y\leq b\land X\neq Y\}\subseteq E^c$.

\paragraph{3.2.2} Suppose for a collection of events indexed by $I$ is independent, such that for any finite choice $\seq{a}{j}\in I$, $P(\subseqconn{A}{a}{j}{\cap})=P(A_{a_1})\dots P(A_{a_j})$.

\subparagraph{a} $P(A_{a_2}\cap\dots\cap A_{a_j}) = P(subseqconn{A}{a}{j}{\cap}) + P(A_{a_1}^c\cap A_{a_2}\cap\dots\cap A_{a_j})$ by additivity.

So we have $P(A_{a_1}^c\cap A_{a_2}\cap\dots\cap A_{a_j}) = P(A_{a_2}\cap\dots\cap A_{a_j}) - P(\subseqconn{A}{a}{j}{\cap}) = P(A_{a_2})\dots P(A_{a_j}) - P(A_{a_1})\dots P(A_{a_j}) = (1-P(A_{a_1})P(A_{a_2})\dots P(A_{a_j}) = P(A_{a_1}^c)P(A_{a_2})\dots P(A_{a_j})$.

\subparagraph{b} Proven by induction, plus associative and commutative properties of $\cap$ and $\cdot$ (product). The idea is to replace each $A_{a_i}, 1\leq i\leq j$ by $A_{a_i}^c$ one by one.

\subparagraph{c} A direct consequence of part (b).

\paragraph{3.2.4} 
\begin{itemize} 
	\item We make a note that if $X, Y$ are random variables, then their joint distribution is always defined, since for Borel set $A, B$, $\{X\in A, Y\in B\} = \{X\in A\}\cap \{Y\in B\}\in alg$, of which $X^{-1}(A), Y^{-1}(B)$ both reside in $\alg$, and $\alg$ is closed under countable intersection.
	\item Another detail is that $P':\borel\rightarrow\real,S\mapsto P(Y\in S)$ is a probability measure sample space $\real$ and the borel algebra on $\real$, $\borel$. To prove this, we first verify that indeed, for each borel set $A\in\borel$, by definition of a r.v., $P'(S) = P(Y^{-1}(S))$ is defined. So, the domain of $P'$ is correct. Next, we show countable additivity. Given disjoint set $\infseq{S}$, we have $P'(\sqcup_i S_i) = P(Y\in \sqcup_i S_i) = P(\sqcup_i \{Y\in S_i\} = \sum_i P(Y\in S_i) = \sum_i P'(S_i)$, where the second last equality holds because $P$ is a probability measure.\\
	Consequently, in the proof of \textbf{proposition 3.2.4}, $Q,P'$ are both probability measures defined on $\borel$, and $Q$ agrees with $P'$ on sets $(-\infty,y]$ and hence the semialgebra $\semialg$ defined in \textbf{2.5.9}, so $Q=P'$ on $\borel$.
\end{itemize}

\end{document}