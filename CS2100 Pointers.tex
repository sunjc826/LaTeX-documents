\documentclass{article}
\usepackage[english]{babel}
\usepackage[utf8]{inputenc}
\usepackage{amsmath,amssymb}
\usepackage{parskip}
\usepackage{graphicx}
\usepackage {tikz}
\usetikzlibrary{arrows.meta}
\usepackage{tabularx,booktabs}
\usepackage{algorithmic}
\newcolumntype{Y}{>{\centering\arraybackslash}X}
\usepackage[shortlabels]{enumitem}
\usepackage{slashed}

\usetikzlibrary {positioning}
%\usepackage {xcolor}
\definecolor {processblue}{cmyk}{0.96,0,0,0}

\newcommand\tab[1][1cm]{\hspace*{#1}}

\usepackage{mathtools}
\DeclarePairedDelimiter\ceil{\lceil}{\rceil}
\DeclarePairedDelimiter\floor{\lfloor}{\rfloor}


\usepackage{hyperref}
\hypersetup{
    colorlinks=true,
    linkcolor=blue,
    filecolor=magenta,      
    urlcolor=cyan,
}


% Margins
\usepackage[top=2.5cm, left=3cm, right=3cm, bottom=4.0cm]{geometry}


\title{CS2100}
\author{Jia Cheng}
\date{January 2021}

\begin{document}
\maketitle

\section{Definitions}


\section{Number representations}
\textbf{Definition} $=$ is the identity relation between 2 mathematical objects.

\textbf{Example} $(x)_{b_1}=(x)_{b_2}$, where $b_1$, $b_2$ are any 2 representations of $x$. If we indicate $b_1,b_2$ by numbers $n,m\in \mathbb{N}$, then such representations could be base-$n/m$ representations respectively.

\textbf{Definition} $\equiv$ is the equivalence relation between 2 object representations.

\textbf{Example} Let $S$ be a set. Let $r$ be a representation of elements of set $S$. Then $(a)_r\equiv (b)_r$ if the \textit{representation} of $a,b$ under $r$ are identical. Note that it is not necessary for $a=b$.

Note that I'm not too sure how a "representation" is defined rigorously.

\subsection{Ones complement}
\textbf{Definition} For this section, let representation $r:=1s$. Then \begin{align*}
	\forall x\in [0, (2^{n-1}-1)]\cap \mathbb{Z}, (x)_r\equiv(x)_2
\end{align*}

\textbf{Definition}  Let $S=[-(2^{n-1}-1), (2^{n-1}-1)]\cap \mathbb{Z}$. Suppose the representation is restricted to $n$ bits. Then \begin{align*}
	\forall x \in [-(2^{n-1}-1), 0)\cap \mathbb{Z}, (-x)_r \equiv (2^n-1-x)_r 
\end{align*}


Note: I have yet to prove these. I think this might be the case.\\

Yet to prove\\
\textbf{Proposition} For all $a,b\in S$, $(a+b)_r
=(a)_r+(b)_r$

Yet to prove\\
\textbf{Proposition} For all $a,b\in S$, $(a+b)_r
\equiv(a)_r+(b)_r$


How to get this without examining the bits?\\
\textbf{Proposition} For negative $x$, $(-x)_r\equiv (2^n-1-x)_r$

\textbf{Proof}  


\textbf{Proposition} $(-(-x))_r\equiv (x)_r$

\textbf{Non-Proof} Trivially, $--x=x$ and hence their representations are the same as well. However, we may wish to verify this purely using the definition of the representation. \\
Then, we have, \begin{align*}
	(-(-x))_r &\equiv (2^n-1-(-x))_r \\
	&\equiv (2^n-1)_r-(-x)_r \\ 
	&\equiv (2^n-1)_r-((2^n-1)_r-(x)_r) \\
	&\equiv ((2^n-1)_r-(2^n-1)_r)+(x)_r \\
	&\equiv (x)_r
\end{align*}


\end{document}
