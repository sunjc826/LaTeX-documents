\documentclass{article}
\usepackage[english]{babel}
\usepackage[utf8]{inputenc}
\usepackage{amsmath,amssymb}
\usepackage{parskip}
\usepackage{graphicx}

\newcommand\tab[1][1cm]{\hspace*{#1}}
\newcommand{\code}{\texttt}

% Margins
\usepackage[top=2.5cm, left=3cm, right=3cm, bottom=4.0cm]{geometry}


\title{Sagemath Quiz Pointers}
\author{Jia Cheng}
\date{November 2020}

\begin{document}

\maketitle

\section*{Quiz Instructions}
\begin{enumerate}
	\item The quiz covers all materials in SageMath Notes Lesson 1-5.
	\item There are 8 sections, and the total mark is 8.
	\item The LumiNUS quiz system will select one question from each section to generate a paper for you.
	\item All questions are fill-in-blank questions, and all answers are positive numbers.
	\item \textbf{If the answer is an integer, fill in that integer.}  For example, the answer of 1+1 is 2.\\
	Note that 2.0 is \textbf{wrong}!
	\item \textbf{If the answer is not an integer, correct to 5 decimal places.}  For example, if the answer is the Euler number e = 2.718281828..., then only numerical answer between 2.71828 and 2.71829 are acceptable.
	\item If the answer is 0.5, since it is not an integer, how can I key in the answer?
The answers like 0.5, 0.50, 0.500, 0.5000, 0.50000 are all acceptable.
\end{enumerate}

\section*{Pointers}
\begin{itemize}
	\item Suggested variables
	\begin{itemize}
		\item summation index i
		\item summation limits m, n
		\item derivative limit h
		\item eval function at values x=a, y=b
		\item function f(x), first derivative g(x), second derivative h(x)
	\end{itemize}
	\item Remember to use var("y") when creating expressions that use y as a variable
	\item Plotting functions
	\begin{itemize}
		\item Plotting 2 single variable functions\\
		\code{plot((f(x), g(x)), (x, left\_limit, right\_limit), ymin=..., ymax=..., plot\_points=..., color=("red", "violet"))}
		\item Plotting 2 implicit functions\\
		\code{A = implicit\_plot(f(x, y), (x, left\_limit, right\_limit), (y, left\_limit, right\_limit))}\\
		\code{B = implicit\_plot(g(x, y), (x, left\_limit, right\_limit), (y, left\_limit, right\_limit))}\\
		\code{A+B}
	\end{itemize}
	\item \code{f(x).derivative(x)}
	\begin{itemize}
		\item This is the variable of differentiation.
	\end{itemize}
	\item \code{f(x, y).implicit\_derivative(y, x)}
	\begin{itemize}
		\item Note that ...(y, x) is for finding dy/dx. So if we want to find dx/dy, then use ...(x, y) as the params.
	\end{itemize}
	\item The right way to use implicit\_derivative
	\begin{itemize}
		\item Suppose $f(x, y) = x^2 + y^2 == 1, g(x, y) = x^2 + y^2 - 1$
		\item f(x, y).implicit\_derivative will cause errors. Instead, do g(x, y).implicit\_derivative
		\item In other words, implicit\_derivative accepts functions instead of equations
	\end{itemize}
	\item Integrals
	\begin{itemize}
		\item Indefinite f(x).integral(x)
		\item Definite f(x).integral(x, a, b)
		\item Algorithms: Default(Maxima), \code{sympy}, \code{mathematica\_free}, \code{giac}\\
		\code{f(x).integral(x, a, b, algorithm="name of algorithm")}\\
		sympy is recommended
		\item When differentiating an integral, use hold=True\\
		As an example, \code{cos(t\^{}2).integral(t, cos(x), 5*x, hold=True).derivative(x).show()}
	\end{itemize}
	\item Generic function
	\begin{itemize}
		\item To declare f as a function in variable t,\\
		\code{var("t")}\\
		\code{function("f")(t)}
	\end{itemize}
	\item For complicated expressions
	\begin{itemize}
		\item \code{.factor()}
		\item \code{.full\_simplify()}
		\item \code{.expand()}
	\end{itemize}
	\item Differential Equations
	\begin{itemize}
		\item \code{function("y")(x)}\\
		\code{desolve(y(x).derivative(x) == x+y(x), y(x))}
		\item With initial conditions\\
		\code{ode = y(x).derivative(x) == x+y(x)\\
		desolve(ode, y(x), ics=[1,2])}\\
		In general, \code{ics=[x,y,dy/dx]}
	\end{itemize}
	\item Evaluation functions
	\begin{itemize}
		\item \code{.find\_root(left\_limit, right\_limit)}\\
		Example: \code{(log(x) == sin(x)).find\_root(1, 3)}
		\item \code{.n(digits=...)}\\
		For numerical approximation
		\item \code{.solve(x)}\\
		Example: \code{(g(x)==x).solve(x, algorithm="sympy")}
		
	\end{itemize}
	\item Other helper functions
	\begin{itemize}
		\item \code{.is\_prime()}
		\item \code{.is\_real()}
	\end{itemize}
\end{itemize}

\end{document}
