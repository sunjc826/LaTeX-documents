\documentclass{article}
\usepackage[english]{babel}
\usepackage[utf8]{inputenc}
\usepackage{amsmath,amssymb}
\usepackage{parskip}
\usepackage{graphicx}

\newcommand\tab[1][1cm]{\hspace*{#1}}

% Margins
\usepackage[top=2.5cm, left=3cm, right=3cm, bottom=4.0cm]{geometry}


\title{MA2108S (Mathematical Analysis I) Pointers}
\author{Jia Cheng}
\date{January 2021}

\begin{document}

\maketitle

\section{Definitions and Formula}

\section{Point Set Topology}
\subsection{Definitions}
\begin{itemize}
	\item Metric Space $M$
	\item Open ball $B(p,r)=\{q\in M : d(p,q) < r \}$
	\item Boundary of S: $bd(S)=\{p\in S : \forall r > 0, \exists q\in S, \exists q'\in S^c, q,q'\in B(p,r) \}$
	\item Limit points of S: $lim(S)=\{p\in S : \forall r > 0, \exists q\in S, q\neq p, q\in B(p,r) \}$
	\item Interior of S: $int(S)=\{p\in S : \exists r > 0, B(p,r)\subseteq S\}$
\end{itemize}
\subsection{Equivalence of definitions of closed set}
The following definitions of a closed set are equivalent.

\begin{enumerate}
	\item $bd(S)\subseteq S$
	\item $lim(S)\subseteq S$
\end{enumerate}

\textbf{Proof}: Suppose $bd(S)\subseteq S$. 
Let $x\in lim(S)$ and fix some arbitrary $r\in \mathbb{R}^+$.
If $x\in S$, we are done.\\
Otherwise,  $x\not \in S$. As $x$ is a limit point, $\exists q\in S, q\neq x$ such that $q\in B(x,r)$. Now, notice that $x$ is a boundary point of $S$, since $x$ itself is not in S, and $q$ is in S, and both $x,q\in B(x,r)$.\\
But by our initial assumption, we have $x\in bd(S)\subseteq S$. Contradiction.\\
Hence, $x\in S$ and $\lim(S)\subseteq S$.

Conversely, suppose $lim(S)\subseteq S$.\\
Let $x\in bd(S)$ and fix some arbitrary $r\in \mathbb{R}^+$. If $x\in S$, we are done.\\
Otherwise, $x\not \in S$. As $x$ is a boundary point, $\exists q\in S, q' \in S^c$ such that both are in $B(x,r)$. In particular, $q\neq x$ since $x\not \in S$. This says that $x$ is a limit point of $S$.\\
Our initial supposition says that $x\in lim(S)\subseteq S$. Again, we have a contradiction.  

\end{document}


