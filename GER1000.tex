\documentclass{article}
\usepackage[english]{babel}
\usepackage[utf8]{inputenc}
\usepackage{amsmath,amssymb}
\usepackage{parskip}
\usepackage{graphicx}

\newcommand\tab[1][1cm]{\hspace*{#1}}

% Margins
\usepackage[top=2.5cm, left=3cm, right=3cm, bottom=4.0cm]{geometry}


\title{GER1000 (Analysis of Statistics)}
\author{Jia Cheng}
\date{January 2021}

\begin{document}

\maketitle

\section{Definitions and Formula}

\section{Misc}
TP-SR5
Seat number 12

\section{Associations}
\subsection{Rates}
Essentially conditional probability.
From here on, define $rate:=P$ (probability function)

\textbf{Definitions and Propositions
}\begin{itemize}
	\item Positive Association of $A$ with $B$: If $P(A|B) > P(A|B^c) \lor P(B|A) > P(B|A^c)$
	\item $P(A|B) = P(A|B^c) \implies P(B|A) = P(B) = P(B|A^c)$
\end{itemize}

\subsection{Groups}
\begin{itemize}
	\item Observational group (non-assigned)
	\item Control Group
	\item Treatment Group
\end{itemize}

\subsection{Confounders}
Confounding variables are associated with both independent and dependent variables.

To reduce confounding, use slicing, so as to compare smaller groups with are relatively homogeneous w.r.t. the factors.

\subsection{Simpson's Paradox}


\subsection{Tutorial 1}
Imagine	that	you	are	an intern	at	a	large	tuition	centre catered	to	students	of	age	11	and	12	
years. Your	 employer	 wants to	 know	 if	 it	 is	 worthwhile	 to	 invest	 in iPads to improve students’	proficiency	in English.	He gives	you	authority	and	resources,	and asks	you	
to design	an	experiment	on	the	thousands	of	customers.

(a)	How	would	you	enrol	subjects	and	assign	them	into	two	groups?

Ask for parental consent, then randomly assign consenting students to control and "treatment" groups.


(b)	How	feasible is	it	to	use	a placebo, or to implement double-blinding?

Placebo is not possible, since you can't just fake an ipad.

Blinding students is difficult, since knowledge of having/not having ipad in lesson is easily known by students and by parents.\\
Single blinding is possible however. For e.g. when testing the students at the end of the trial, do not inform the accessors about which group they belong to.




\end{document}




