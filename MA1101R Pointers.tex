\documentclass{article}
\usepackage[english]{babel}
\usepackage[utf8]{inputenc}
\usepackage{amsmath,amssymb}
\usepackage{parskip}
\usepackage{graphicx}

\newcommand\tab[1][1cm]{\hspace*{#1}}

% Margins
\usepackage[top=2.5cm, left=3cm, right=3cm, bottom=4.0cm]{geometry}


\title{MA1101R}
\author{Jia Cheng}
\date{September 2020}

\begin{document}

\maketitle

\paragraph{Uniqueness of row-reduced echelon forms} We provide a proof for the uniqueness of RREFs.

Let $A$ be an augmented $m\times (n+1)$ matrix representing a linear system $S$ in $n$ variables. Let $R_1, R_2$ be 2 matrices in RREF equivalent to $A$. (Here equivalence is known as row equivalence.) By symmetry and transitivity of equivalence, we have $R_1\equiv R_2$.

Claim: The pivot (and non-pivot) columns of $R_1, R_2$ are the same.

Suppose not. WLOG (if not we simply swap the roles of $R_1$ and $R_2$), there exists an index $i\in \{1,2,\dots, n\}$ such that the $i$-th column is a pivot column in $R_2$ but not a pivot in $R_1$. We then form a new $2m\times (n+1)$ matrix by stacking $R_2$ below $R_1$. \\
We note that the linear system $S'$ represented by this stacked matrix is equivalent to the original linear system $S$ because: Let $S_1$ be the linear system described by $R_1$, $S_2$ be the linear system described by $R_2$. Then $S' = S_1\land S_2 = S\land S = S$.\\
By performing Gauss-Jordan elimination again, we see that the resulting matrix consists of strictly more pivot columns than $R_1$ originally had. In particular, all the pivot columns of $R_1$ are still pivot columns, plus column $i$ and possibly some more columns. This is a contradiction since the general set of solutions had a strictly smaller dimension. Even if we ignore dimension, we can make the following argument. Let $C\subseteq \{1,2,\dots,n\}$ be the indices of non-pivot columns in $R_1$, $C'$ be the  number of non-pivot columns in the RREF of the stacked matrix. Then $C'\subsetneq C$.

For any fixed tuple corresponding to the non-pivot columns of $C'$, there are infinitely many solutions to be found by $S_1$, since $C-C'\neq \emptyset$ and we can vary any variable corresponding to the non-pivot columns in $C-C'$. Whereas in system $S'$, fixed the variables corresponding to $C'$ forces a single tuple of solutions. This suggests that $S_1$ and $S$ are not equivalent, which is a contradiction.

Hence $R_1, R_2$ have the same pivot columns. Now we consider the first row of $R_1, R_2$.

Claim: The first row of these 2 matrices must be the same.

We already know that cells in other pivot columns must be zero for both matrices, by definition of RREF. Now consider the cells corresponding to the non-pivot columns. We see that the variable corresponding to the pivot cell can be expressed in terms of the variables corresponding to the non-pivot cells. In other words, for both $R_1, R_2$, the coefficients w.r.t. those non-pivot variables must be the same. This says that all cells in the first row of $R_1, R_2$ must be the same.

By repeating this argument for every row, we prove that all rows of $R_1, R_2$ are equal and hence $R_1 = R_2$.

\end{document}
