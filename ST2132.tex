\documentclass[a4paper]{article}
\usepackage[english]{babel}
\usepackage[utf8]{inputenc}
\usepackage{amsmath,amssymb}
\usepackage{parskip}
\usepackage{graphicx}
\graphicspath{ {./images/} }
\usepackage{gensymb}
\usepackage{listings}
\usepackage{caption}
\usepackage{pgfplots}

\newcommand\tab[1][1cm]{\hspace*{#1}}
\newcommand{\set}[1]{\{#1\}}
\newcommand{\rbracket}[1]{\left(#1\right)}
\newcommand{\sbracket}[1]{\left[#1\right]}
\newcommand{\powerset}[1]{\mathcal{P}(#1)}
\newcommand{\reals}[0]{\mathbb{R}} %real numbers%
\newcommand{\real}[0]{\mathbb{R}} %real numbers%
\newcommand{\rational}[0]{\mathbb{Q}} %rational numbers%
\newcommand{\integer}[0]{\mathbb{Z}} %integers%
\newcommand{\nat}[0]{\mathbb{N}} %natural numbers%
\newcommand{\limitinf}[1][n]{\lim_{#1\rightarrow \infty}} %limit to infinity%
\newcommand{\forallReal}[1][x]{\forall #1\in\real} 
\newcommand{\forallInteger}[1][n]{\forall #1\in\integer}
\newcommand{\forallNat}[1][n]{\forall #1\in\nat}
\newcommand{\existsReal}[1][x]{\exists #1\in\real}
\newcommand{\existsInteger}[1][n]{\exists #1\in\integer}
\newcommand{\existsNat}[1][n]{\exists #1\in\nat}
\newcommand{\enum}[1]{1,2,\dots,#1}
\newcommand{\seq}[2]{#1_1,#1_2,\dots,#1_{#2}} %finite sequence literal%
\newcommand{\infseq}[1]{#1_1,#1_2,\dots} %infinite sequence%
\newcommand{\subseq}[3]{#1_{#2_1},#1_{#2_2},\dots,#1_{#2_{#3}}} %finite subsequence literal%
\newcommand{\seqconn}[3]{#1_1 #3 #1_2 #3 \dots #3 #1_{#2}} %sequence delimited by argument 3%
\newcommand{\subseqconn}[4]{#1_{#2_1} #4 #1_{#2_2} #4 \dots #4 #1_{#2_{#3}}} %subsequence literal%

\newcommand{\io}[1]{#1 \mathit{i.o.}} %infinitely often%
\newcommand{\alma}[1]{#1 \mathit{a.a.}} %almost always%

\newcommand{\qedsymbol}{\hfill\blacksquare}

\newcommand{\floor}[1]{\lfloor #1\rfloor}

% \rule{length}{thickness}
\newcommand{\divider}[0]{\begin{center}
\rule{16cm}{0.5pt}
\end{center}}

\newcommand{\weakconv}{\rightarrow_P}

% Margins
% top=2.54cm, left=2.54cm, right=2.54cm, bottom=2.54cm
\usepackage[a4paper, top=2.54cm, left=2.54cm, right=2.54cm, bottom=2.54cm]{geometry}
\usepackage{hyperref}
\hypersetup{
    colorlinks=true,
    linkcolor=blue,
    filecolor=magenta,      
    urlcolor=blue,
}

\title{ST2132}
\author{Jia Cheng}
\date{Jan 2022}

\begin{document}
\maketitle

\section{Limit Theorems}
\paragraph{Properties of weak convergence} (Also known as convergence in probability)

\subparagraph{Additivity} Suppose $A_n\weakconv \alpha, B_n\weakconv \beta$ for sequences of random variables $(A_n), (B_n)$. Then,
\begin{align*}
	P(|(A_n + B_n) - (\alpha + \beta)| > \epsilon) \leq P(|A_n-\alpha| + |B_n+\beta| > \epsilon) \leq P(\set{|A_n-\alpha| > \frac{\epsilon}{3}} \cup \set{|B_n-\beta| > \frac{\epsilon}{3}})\rightarrow 0 + 0 = 0
\end{align*}
Therefore, $A_n + B_n\weakconv \alpha + \beta$.\\
Here, we use the inequalities $|A_n-\alpha| + |B_n-\beta| \geq |A_n-\alpha+B_n-\beta|$ and $\epsilon > \frac{\epsilon}{3} + \frac{\epsilon}{3}$.

\subparagraph{Closure under Continuity} Suppose $X_n\weakconv \alpha$. Let $g$ be a continuous function. Then in particular, $g$ is continuous at $\alpha$.

For a fixed $\epsilon$, let $\delta > 0$ be such that $|x-\alpha|\leq\delta\implies |g(x)-g(\alpha)|\leq\epsilon$. The converse then says that $|g(x)-g(\alpha)| > \epsilon\implies |x-\alpha| > \delta$, which is what we will use below.
\begin{align*}
	P(|g(X_n)-g(\alpha)| > \epsilon)\leq P(|X_n-\alpha| > \delta)\rightarrow 0
\end{align*}
Hence, $g(X_n)\weakconv g(\alpha)$.

\section{Distributions derived from Normal distribution}

\subparagraph{t-distribution pdf derivation}
First, note that $T=\frac{Z}{\sqrt{\frac{U}{n}}}\sim t_n$.\\
Let $Y = Z, X = \sqrt{\frac{U}{n}}$.

\begin{align*}
	F_{X}(u) = P(\sqrt{\frac{U}{n}}\leq u) = P(U\leq nu^2) = F_U(nu^2)
\end{align*}
Hence,
\begin{align*}
	f_{X}(u) = 2nu f_U(u) &= 2nu\cdot \frac{\rbracket{\frac{1}{2}}^{\frac{n}{2}}}{\Gamma\rbracket{\frac{n}{2}}} (nu^2)^{\frac{n}{2}-1}e^{-\frac{nu^2}{2}} [nu^2\geq 0]\\
	&= \frac{n^{\frac{n}{2}}}{2^{\frac{n}{2}-1}\Gamma\rbracket{\frac{n}{2}}}u^{n-1}e^{-\frac{nu^2}{2}}[u\geq 0]
\end{align*}
Note the formula for the pdf of a quotient.
\begin{align*}
	f_{\frac{Y}{X}}(t) = \int_\real |x|f_X(x)f_Y(tx)dx
\end{align*}
Hence,
\begin{align*}
	f_T(t) &= \int_\real |x|\frac{n^{\frac{n}{2}}}{2^{\frac{n}{2}-1}\Gamma\rbracket{\frac{n}{2}}}x^{n-1}e^{-\frac{nt^2}{2}}[t\geq 0]\cdot \frac{1}{\sqrt{2\pi}}e^{-\frac{(tx)^2}{2}}dx\\
	&= \frac{n^{\frac{n+1}{2}}}{2^{\frac{n-1}{2}}\sqrt{n\pi}\Gamma\rbracket{\frac{n}{2}}} \int_{\real^+} x^ne^{\frac{-(n+t^2)x^2}{2}}dx\\
	&= \frac{n^{\frac{n+1}{2}}}{2^{\frac{n-1}{2}}\sqrt{n\pi}\Gamma\rbracket{\frac{n}{2}}} \int_{\real^+} \rbracket{\frac{2}{n+t^2}u}^{\frac{n}{2}}e^{-u}\frac{1}{\sqrt{n+t^2}\sqrt{2u}}du \text{ reverse substitution: } x=\sqrt{\frac{2}{n+t^2}u}\\
	&=\frac{n^{\frac{n+1}{2}}}{2^{\frac{n-1}{2}}\sqrt{n\pi}\Gamma\rbracket{\frac{n}{2}}}\cdot \frac{2^{\frac{n-1}{2}}}{(n+t^2)^{\frac{n+1}{2}}}\int_{\real^+}u^{\frac{n-1}{2}}e^{-u}du\\
	&=\frac{1}{\Gamma\rbracket{\frac{n}{2}}\sqrt{n\pi}}\rbracket{\frac{n+t^2}{n}}^{-\frac{n+1}{2}}\int_{\real^+}u^{\frac{n+1}{2}-1}e^{-u}du\\
	&=\frac{\Gamma\rbracket{\frac{n+1}{2}}}{\Gamma\rbracket{\frac{n}{2}}\sqrt{n\pi}}\rbracket{\frac{n+t^2}{n}}^{-\frac{n+1}{2}}
\end{align*}

\paragraph{Square of t-distribution}
\begin{align*}
	T^2 = \frac{Z^2}{\frac{U}{n}}
\end{align*}
where $Z^2\sim \chi_1^2, U\sim \chi_n^2$, hence $T^2\sim F_{1,n}$

\paragraph{F-distribution pdf derivation}
Note that
\begin{align*}
	W = \frac{\frac{U}{m}}{\frac{V}{n}}
\end{align*}
where $U\sim \chi_m^2, V\sim \chi_n^2$.

We can derive that
\begin{align*}
	F_{\frac{U}{m}}(x)=F_U(mx)\implies f_{\frac{U}{m}}(x) = mf_U(mx)\\
	F_{\frac{V}{n}}(x)=F_V(nx)\implies f_{\frac{V}{n}}(x) = nf_V(nx)
\end{align*}

\begin{align*}
	f_W(w) &= \int_\real |x|f_X(x)f_Y(wx)dx\\
	&= \int_\real |x|\cdot nf_V(nx)\cdot mf_U(mwx)dx\\
	&= \int_\real |x|\cdot mn\cdot [nx\geq 0]\frac{\rbracket{\frac{1}{2}}^{\frac{n}{2}}}{\Gamma\rbracket{\frac{n}{2}}}(nx)^{\frac{n}{2}-1}e^{-\frac{nx}{2}}\cdot [mwx\geq 0]\frac{\rbracket{\frac{1}{2}}^{\frac{m}{2}}}{\Gamma\rbracket{\frac{m}{2}}}(mwx)^{\frac{m}{2}-1}e^{-\frac{mwx}{2}}
\end{align*}
We consider cases. 
\begin{itemize}
	\item If $w\leq 0$, then $[nx\geq 0][mwx\geq 0] = [x\geq 0][x\leq 0] = [x=0]$, so the integral is over a single point, hence evaluates to $0$.
	\item Otherwise, if $w>0$, then $[nx\geq 0][mwx\geq 0] = [x\geq 0]$, so the integral can then be restricted to $\mathbb{R}^+$.
\end{itemize}
Hence, in the case where $w > 0$, we need to evaluate
\begin{align*}
	f_W(w) &= \int_{\real^+} x\cdot mn\cdot \frac{\rbracket{\frac{1}{2}}^{\frac{n}{2}}}{\Gamma\rbracket{\frac{n}{2}}}(nx)^{\frac{n}{2}-1}e^{-\frac{nx}{2}}\cdot \frac{\rbracket{\frac{1}{2}}^{\frac{m}{2}}}{\Gamma\rbracket{\frac{m}{2}}}(mwx)^{\frac{m}{2}-1}e^{-\frac{mwx}{2}}\\
	&= \frac{\rbracket{\frac{1}{2}}^{\frac{m+n}{2}}(m)^{\frac{m}{2}}w^{\frac{m}{2}-1}n^{\frac{n}{2}}}{\Gamma\rbracket{\frac{m}{2}}\Gamma\rbracket{\frac{n}{2}}}\int_{\real^+} x^{\frac{m+n}{2}-1}e^{-\frac{n+mw}{2}x}dx\\
	&= \frac{\rbracket{\frac{1}{2}}^{\frac{m+n}{2}}\rbracket{\frac{m}{n}}^{\frac{m}{2}}w^{\frac{m}{2}-1}n^{\frac{m+n}{2}}}{\Gamma\rbracket{\frac{m}{2}}\Gamma\rbracket{\frac{n}{2}}}\int_{\real^+} x^{\frac{m+n}{2}-1}e^{-\frac{n+mw}{2}x}dx\\
	&= \frac{\rbracket{\frac{1}{2}}^{\frac{m+n}{2}}\rbracket{\frac{m}{n}}^{\frac{m}{2}}w^{\frac{m}{2}-1}n^{\frac{m+n}{2}}}{\Gamma\rbracket{\frac{m}{2}}\Gamma\rbracket{\frac{n}{2}}}\int_{\real^+} \rbracket{\frac{2}{n+mw}u}^{\frac{m+n}{2}-1}e^{-u}\frac{2}{n+mw}dx \text{ Substitution: }u=\frac{n+mw}{2}x\\
	&= \frac{\rbracket{\frac{1}{2}}^{\frac{m+n}{2}}\rbracket{\frac{m}{n}}^{\frac{m}{2}}w^{\frac{m}{2}-1}n^{\frac{m+n}{2}}}{\Gamma\rbracket{\frac{m}{2}}\Gamma\rbracket{\frac{n}{2}}}\int_{\real^+} \rbracket{\frac{2}{n+mw}u}^{\frac{m+n}{2}-1}e^{-u}\frac{2}{n+mw}dx\\
	&= \frac{\rbracket{\frac{m}{n}}^{\frac{m}{2}}w^{\frac{m}{2}-1}n^{\frac{m+n}{2}}}{\Gamma\rbracket{\frac{m}{2}}\Gamma\rbracket{\frac{n}{2}}}\int_{\real^+} \rbracket{\frac{1}{n+mw}u}^{\frac{m+n}{2}-1}e^{-u}\frac{1}{n+mw}dx\\
	&= \frac{\rbracket{\frac{m}{n}}^{\frac{m}{2}}w^{\frac{m}{2}-1}}{\Gamma\rbracket{\frac{m}{2}}\Gamma\rbracket{\frac{n}{2}}}\rbracket{\frac{n+mw}{n}}^{-\frac{m+n}{2}}\int_{\real^+} u^{\frac{m+n}{2}-1}e^{-u}dx\\
	&= \frac{\Gamma\rbracket{\frac{m+n}{2}}}{\Gamma\rbracket{\frac{m}{2}}\Gamma\rbracket{\frac{n}{2}}}\rbracket{\frac{m}{n}}^{\frac{m}{2}}w^{\frac{m}{2}-1}\rbracket{1+\frac{m}{n}w}^{-\frac{m+n}{2}}
\end{align*}

\end{document}