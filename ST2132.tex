\documentclass[a4paper]{article}
\usepackage[english]{babel}
\usepackage[utf8]{inputenc}
\usepackage{amsmath,amssymb}
\usepackage{parskip}
\usepackage{graphicx}
\graphicspath{ {./images/} }
\usepackage{gensymb}
\usepackage{listings}
\usepackage{caption}
\usepackage{pgfplots}

\newcommand\tab[1][1cm]{\hspace*{#1}}
\newcommand{\set}[1]{\{#1\}}
\newcommand{\powerset}[1]{\mathcal{P}(#1)}
\newcommand{\reals}[0]{\mathbb{R}} %real numbers%
\newcommand{\real}[0]{\mathbb{R}} %real numbers%
\newcommand{\rational}[0]{\mathbb{Q}} %rational numbers%
\newcommand{\integer}[0]{\mathbb{Z}} %integers%
\newcommand{\nat}[0]{\mathbb{N}} %natural numbers%
\newcommand{\limitinf}[1][n]{\lim_{#1\rightarrow \infty}} %limit to infinity%
\newcommand{\forallReal}[1][x]{\forall #1\in\real} 
\newcommand{\forallInteger}[1][n]{\forall #1\in\integer}
\newcommand{\forallNat}[1][n]{\forall #1\in\nat}
\newcommand{\existsReal}[1][x]{\exists #1\in\real}
\newcommand{\existsInteger}[1][n]{\exists #1\in\integer}
\newcommand{\existsNat}[1][n]{\exists #1\in\nat}
\newcommand{\enum}[1]{1,2,\dots,#1}
\newcommand{\seq}[2]{#1_1,#1_2,\dots,#1_{#2}} %finite sequence literal%
\newcommand{\infseq}[1]{#1_1,#1_2,\dots} %infinite sequence%
\newcommand{\subseq}[3]{#1_{#2_1},#1_{#2_2},\dots,#1_{#2_{#3}}} %finite subsequence literal%
\newcommand{\seqconn}[3]{#1_1 #3 #1_2 #3 \dots #3 #1_{#2}} %sequence delimited by argument 3%
\newcommand{\subseqconn}[4]{#1_{#2_1} #4 #1_{#2_2} #4 \dots #4 #1_{#2_{#3}}} %subsequence literal%

\newcommand{\io}[1]{#1 \mathit{i.o.}} %infinitely often%
\newcommand{\alma}[1]{#1 \mathit{a.a.}} %almost always%

\newcommand{\qedsymbol}{\hfill\blacksquare}

\newcommand{\floor}[1]{\lfloor #1\rfloor}

% \rule{length}{thickness}
\newcommand{\divider}[0]{\begin{center}
\rule{16cm}{0.5pt}
\end{center}}

\newcommand{\weakconv}{\rightarrow_P}

% Margins
% top=2.54cm, left=2.54cm, right=2.54cm, bottom=2.54cm
\usepackage[a4paper, top=2.54cm, left=2.54cm, right=2.54cm, bottom=2.54cm]{geometry}
\usepackage{hyperref}
\hypersetup{
    colorlinks=true,
    linkcolor=blue,
    filecolor=magenta,      
    urlcolor=blue,
}

\title{ST2132}
\author{Jia Cheng}
\date{Jan 2022}

\begin{document}
\maketitle

\section{Limit Theorems}
\paragraph{Properties of weak convergence} (Also known as convergence in probability)

\subparagraph{Additivity} Suppose $A_n\weakconv \alpha, B_n\weakconv \beta$ for sequences of random variables $(A_n), (B_n)$. Then,
\begin{align*}
	P(|(A_n + B_n) - (\alpha + \beta)| > \epsilon) \leq P(|A_n-\alpha| + |B_n+\beta| > \epsilon) \leq P(\set{|A_n-\alpha| > \frac{\epsilon}{3}} \cup \set{|B_n-\beta| > \frac{\epsilon}{3}})\rightarrow 0 + 0 = 0
\end{align*}
Therefore, $A_n + B_n\weakconv \alpha + \beta$.\\
Here, we use the inequalities $|A_n-\alpha| + |B_n-\beta| \geq |A_n-\alpha+B_n-\beta|$ and $\epsilon > \frac{\epsilon}{3} + \frac{\epsilon}{3}$.

\subparagraph{Closure under Continuity} Suppose $X_n\weakconv \alpha$. Let $g$ be a continuous function. Then in particular, $g$ is continuous at $\alpha$.

For a fixed $\epsilon$, let $\delta > 0$ be such that $|x-\alpha|\leq\delta\implies |g(x)-g(\alpha)|\leq\epsilon$. The converse then says that $|g(x)-g(\alpha)| > \epsilon\implies |x-\alpha| > \delta$, which is what we will use below.
\begin{align*}
	P(|g(X_n)-g(\alpha)| > \epsilon)\leq P(|X_n-\alpha| > \delta)\rightarrow 0
\end{align*}
Hence, $g(X_n)\weakconv g(\alpha)$.

\end{document}