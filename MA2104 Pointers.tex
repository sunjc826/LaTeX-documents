\documentclass{article}
\usepackage[english]{babel}
\usepackage[utf8]{inputenc}
\usepackage{amsmath,amssymb}
\usepackage{parskip}
\usepackage{graphicx}

\newcommand\tab[1][1cm]{\hspace*{#1}}

% Margins
\usepackage[top=2.5cm, left=3cm, right=3cm, bottom=4.0cm]{geometry}


\title{MA2104 (Multivariable Calculus) Pointers}
\author{Jia Cheng}
\date{September 2020}

\begin{document}

\maketitle

\section{Definitions and Formula}


\section{Vectors, Lines and Planes}
\subsection{The dot product}
Here, I reference MA1101R's textbook, (Linear Algebra, Concepts and Techniques in Euclidean Space).
Theorems in geometry and trigonometry gives rise to the equalities
\begin{align*}
	\cos(\theta) &= \frac{x_1y_1+x_2y_2}{|\mathbf{x}||\mathbf{y}|} &&\text{in } \mathbb{R}^2\\
	\cos(\theta) &= \frac{x_1y_1+x_2y_2+x_3y_3}{|\mathbf{x}||\mathbf{y}|} &&\text{in } \mathbb{R}^3
\end{align*} 
Hence, we define the dot product in $\mathbb{R}^2$ and $\mathbb{R}^3$. Now, we extend this definition to $\mathbb{R}^n$, such that $\mathbf{x}\cdot \mathbf{y} = \sum_{1\leq i\leq n}x_iy_i$. We also extend the definition of angle to $\mathbb{R}^n$, such that 
\begin{align*}
	\cos(\theta) &= \frac{\mathbf{x}\cdot \mathbf{y}}{|\mathbf{x}||\mathbf{y}|}
\end{align*}
By the Cauchy Schwarz inequality, it turns out that such a generalisation is consistent, as $\frac{\mathbf{x}\cdot \mathbf{y}}{|\mathbf{x}||\mathbf{y}|}$ is always bounded between $-1$ and $1$.

Note that the angle $\theta$ between vectors is by definition the smaller angle, i.e. $0\leq \theta \leq \pi$.


\end{document}


